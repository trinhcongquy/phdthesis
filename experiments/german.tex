The directory-based cache protocol consists of a central server $H$
(for \emph{Home}), and a set of client caches $C_1,\ldots,C_n$. Each
client $C_i$ communicates with the server through a set of three
message channels namely $ch_1^i,ch_2^i,ch_3^i$ in a star
topology.

A client $C_i$ can request to share a given cache and sends a
\emph{ReqShared} or \emph{ReqExclusive} through its $ch_1^i$. The
\emph{Home} node will reply in $ch_2^i$ with an \emph{Invalidate}, a
\emph{GrantShared}, or a \emph{GrantExclusive} message. The client
aknowledges the reception by sending a \emph{InvAck} message in
$ch_3^i$.

A cache line can be in state \emph{Invalid}, \emph{Shared} or
\emph{Exclusive}. If a cache line is in state \emph{Invalid} in the
cache $C_i$, the client $i$ does not have access to that cache line.
If a client has been granted the access, by \emph{Home}, possibly
along with other clients, the cache line is in state \emph{Shared}.
\emph{Home} can also grant the access exclusively to a client, if
which case the cache line is in state \emph{Exclusive}.

\begin{wrapfigure}{r}[2pt]{0.45\textwidth}
  \hfill
  \begin{tikzpicture}[auto,>=stealth']
    \node[scale=0.7,ellipse,ball color=blue!50!white](home){Home};
    \foreach \i/\t/\a in {1/$C_1$/60, 2/$C_2$/120, 3/$C_3$/180, 4/$C_4$/240, 5/$\ldots$/-80, 6/$\ldots$/-40, 7/$C_6$/360}{
      \node(c\i)[scale=0.8,circle,ball color=green!50!white] at +(\a:23mm){\t};
      \begin{pgfonlayer}{background}
        \draw[->] (home.\a) .. controls +(\a:1cm) and +(\a:-1cm) .. (c\i) coordinate[midway](ch2\i);
        \draw[<-] ([rotate=5]home.\a) .. controls +([rotate=5]\a:1cm) and +(\a:-1cm) .. ([rotate=5]c\i) coordinate[pos=0.2](ch1\i);
        \draw[<-] ([rotate=-5]home.\a) .. controls +([rotate=-5]\a:1cm) and +(\a:-1cm) .. ([rotate=-5]c\i) coordinate[pos=0.8](ch3\i);
      \end{pgfonlayer}
      \node[scale=0.4,anchor=east,inner sep=0pt, fill=white!90!gray] at (ch11){$ch_1$};
      \node[scale=0.4,anchor=center,inner sep=0pt, fill=white!90!gray] at (ch21){$ch_2$};
      \node[scale=0.4,anchor=north west,inner sep=0pt, fill=white!90!gray] at (ch31){$ch_3$};
    }
  \end{tikzpicture}
\end{wrapfigure}
%
Initially, all channels are empty and all cache lines are in state
\emph{Invalid}. A bad configuration is detected when two (or more)
processes have exclusive access to a given cache line, or if one
accesses the cache line exclusively whilst the others still have a
shared access.

We model each client in a parametrized system with multiset topology.
The actions of $H$ are represented in each process and its (bounded)
local variables are modeled as shared variables. The model
follows~\cite{PRZ-tacas01}. We assume the channels to be of length one
and therefore can represent them by a local variable (for each
client).  In addition to channels, the central controller manipulates
five data:
\begin{description}[leftmargin=6em,style=nextline,align=right,labelsep=\parindent]
\item[\color{blue} excl] a flag to remember whether the exclusive access has been granted.% (modeled with a shared boolean variable).
\item[\color{blue} ctl] a pointer to the client that sent the request being served.% (modeled with a local boolean variable),
\item[\color{blue} sh\_list] a list of the processes having an access, either shared or exclusive, to the cache~line. % (modeled with a local boolean variable), and
\item[\color{blue} inv\_list] a list of  processes which have to be invalidated in order to serve the current request.% (modeled also with a local boolean variable). 
\item[\color{blue} cmd] a message read in some buffer.% (modeled as shared variable). 
\end{description}%
% We model a variable of the server as follows: Each client has a local
% copy of it. When the server performs an update to that shared
% variable, a broadcast communication is issued: all clients update
% their local copy simultaneously.

A client $i$ may perform one of the following actions:
\begin{description}[leftmargin=6em,style=nextline,align=right,labelsep=\parindent]
\item[\color{orange} $c_1$] If in state invalid and $ch_1^i$ is empty, the client $i$ sends a request to $H$
  for a shared access via its $ch_1$.
\item[\color{orange} $c_2$] If in state invalid or shared while $ch_1^i$ is empty, the client $i$ sends 
  a request to $H$ for an exclusive access via its $ch_1$.
\item[\color{orange} $c_3$] If the client $i$ is granted a shared access via $ch_2^i$,
  it consumes the message from the channel and update the state of the
  cache line to \emph{Shared}. 
\item[\color{orange} $c_4$] If the client $i$ is granted an exclusive access via $ch_2^i$,
  it consumes the message from the channel and update the state of the
  cache line to \emph{Exclusive}.
\item[\color{orange} $c_5$] If the client $i$ receives an invalidation message through
  $ch_2^i$ and $ch_3^i$ is empty, it changes the state of the cache
  line to \emph{Invalid}, empties $ch_2^i$ and sends an invalidation
  acknowledgment to $H$ via $ch_3^i$.
\end{description}

Depending on the content of the channels and the values of the shared
variables, $H$ may perform one of the following actions.
\begin{description}[leftmargin=6em,style=nextline,align=right,labelsep=\parindent]
\item[\color{orange} $h_1$] If $H$ is idle (ie $cmd$ is empty) and receives a request via
  some $ch_1$, it consumes the received request from $ch_1$ into
  $cmd$, selects the sender to be the current client and copies the
  content of the sharer list to the invalidation list.
\item[\color{orange} $h_2$] In case some $ch_2$ is empty, the current command is a
  shared request and the exclusive access has not been granted, then
  $H$ grants a shared access to the current client via $ch_2$ and adds
  the client to the shared list and returns idle.
\item[{\color{orange} $h_3$},{\color{orange}~$h_4$}] If the current command is either a shared request
  (while the exclusive flag is set) or an exclusive request, some
  $ch_2$ is empty, then $H$ sends an invalidation message every
  process through $ch_2$ and removes these processes from the
  invalidation list.
\item[\color{orange} $h_5$] In case the current command is a request for either a
  shared or an exclusive access and $H$ receives an invalidation
  acknowledgment from a client via $ch_3$, then $H$ removes a client
  from the sharer list, resets the exclusive flag and empties $ch_3$.
\end{description}
%
Using this model, the safety properties we checked are: (i) no two
clients are simultaneously granted an exclusive access, and (ii) no
client in state shared coexists with a client in state exclusive.
%
We described the model, the variables and transitions in plain
text. The complete description and pseudocode can be found
in~\cite{PRZ-tacas01}. 

%% ================================================================
\subsection*{Modeled as a Petri Net.}
A simplified version of the protocol has been modeled as a Petri Net
as in~\cite{Raskin:experiments:German}. It uses 12 places:
\emph{idle,serveS,serveE,grantS,} and \emph{grantE} model which
request is being served, \emph{ex,notEx} to model the exclusive flag,
\emph{waitS,waitE} to model the response from $H$, and finally
\emph{shared,excl,invalid} to model the cache line states.


{\noindent\centering %
\begin{tikzpicture}[show background rectangle,
  node distance=6mm,
  label distance=2pt,
  ]

  %% (new R.apply ~descr:"reqS" [idle;invalid] [serveS;waitS]);
  \begin{scope}[xshift=-40mm]
    \node[transition](t){};
    \node[state,place,label={[scale=0.5]above:$idle$},above=of t.west](i1){};
    \node[state,place,label={[scale=0.5]above:$inv$},above=of t.east](i2) {};
    \node[state,place,label={[scale=0.5]below:$serveS$},below=of t.west](o1) {};
    \node[state,place,label={[scale=0.5]below:$waitS$},below=of t.east](o2) {};
    \draw[myedge,<-] (t.150) .. controls +(up:5mm) .. (i1);
    \draw[myedge,<-] (t.30) .. controls +(up:5mm) .. (i2);
    \draw[myedge,->] (t.-150) .. controls +(down:5mm) .. (o1);
    \draw[myedge,->] (t.-30) .. controls +(down:5mm) .. (o2);
  \end{scope}

  %% (new R.apply ~descr:"invalid" [serveS;ex;excl] [notEx;grantS;invalid]);
  \begin{scope}
    \node[transition](t){};
    \node[state,place,label={[scale=0.5]above:$shared$},above=of t.center](i2) {};
    \node[state,place,label={[scale=0.5]above:$serveE$},left=of i2](i1) {};
    \node[state,place,label={[scale=0.5]above:$excl$},right=of i2](i3) {};
    \node[state,place,label={[scale=0.5]below:$grantE$},below=of t.west](o1) {};
    \node[state,place,label={[scale=0.5]below:$inv$},below=of t.east](o2) {};
    \draw[myedge,<-] (t.150) .. controls +(up:5mm) .. (i1);
    \draw[myedge,<-,decorate] (t.30) .. controls +(up:5mm) .. (i2);
    \draw[myedge,<-,mysnake] (t.15) .. controls +(75:2mm) .. (i3);
    \draw[myedge,->] (t.-150) .. controls +(down:5mm) .. (o1);
    \draw[myedge,->,mysnake] (t.-30) .. controls +(-75:5mm) .. (o2);
  \end{scope}

  %% (new R.apply ~descr:"nonex" [serveS;notEx] [grantS;notEx]);
  \begin{scope}[xshift=40mm]
    \node[transition](t){};
    \node[state,place,label={[scale=0.5]above:$serveS$},above=of t.west](i1) {};
    \node[state,place,label={[scale=0.5]right:$notEx$},right=of t](notex) {};
    \node[state,place,label={[scale=0.5]below:$grantS$},below=of t.west](o1) {};
    \draw[myedge,<-] (t.150) .. controls +(up:5mm) .. (i1);
    \draw[myedge,<-] (t.30) .. controls +(up:5mm) and +(120:5mm) .. (notex);
    \draw[myedge,->] (t.-150) .. controls +(down:5mm) .. (o1);
    \draw[myedge,->] (t.-30) .. controls +(down:5mm) and +(-120:5mm) .. (notex);
  \end{scope}

  %% (new R.apply ~descr:"grantS" [grantS;waitS] [idle;shared]);
  \begin{scope}[xshift=-40mm,yshift=-35mm]
    \node[transition](t){};
    \node[state,place,label={[scale=0.5]above:$grantS$},above=of t.west](i1) {};
    \node[state,place,label={[scale=0.5]above:$waitS$},above=of t.east](i2) {};
    \node[state,place,label={[scale=0.5]below:$idle$},below=of t.west](o1) {};
    \node[state,place,label={[scale=0.5]below:$shared$},below=of t.east](o2) {};
    \draw[myedge,<-] (t.150) .. controls +(up:5mm) .. (i1);
    \draw[myedge,<-] (t.30) .. controls +(up:5mm) .. (i2);
    \draw[myedge,->] (t.-150) .. controls +(down:5mm) .. (o1);
    \draw[myedge,->] (t.-30) .. controls +(down:5mm) .. (o2);
  \end{scope}

  %% (new R.apply ~descr:"reqE" [idle;invalid] [serveE;waitE]);
  %% (new R.apply ~descr:"reqE2" [idle;shared] [serveE;waitE]);
  \begin{scope}[yshift=-26mm]% magic number
    \node[state,place,label={[scale=0.5]above:$idle$}](idle){};
    \node[state,place,label={[scale=0.5]above:$shared$},left=of idle](shared) {};
    \node[state,place,label={[scale=0.5]above:$inv$},right=of idle](inv) {};

    \node[transition,minimum width=7mm,below=of shared,anchor=west](t1){};
    \node[transition,minimum width=7mm,below=of inv,anchor=east](t2){};

    \node[state,place,label={[scale=0.5]below:$serveE$},below=of t1](serveE) {};
    \node[state,place,label={[scale=0.5]below:$waitE$},below=of t2](waitE) {};

    \draw[myedge,<-] (t1.150) .. controls +(up:2mm) .. (shared);
    \draw[myedge,<-] (t1.30) .. controls +(up:2mm) .. (idle);
    \draw[myedge,->] (t1.-150) .. controls +(down:2mm) .. (serveE.60);
    \draw[myedge,->] (t1.-30) .. controls +(down:2mm) .. (waitE.120);

    \draw[myedge,<-] (t2.150) .. controls +(up:2mm) .. (idle);
    \draw[myedge,<-] (t2.30) .. controls +(up:2mm) .. (inv);
    \draw[myedge,->] (t2.-150) .. controls +(down:2mm) .. (serveE.60);
    \draw[myedge,->] (t2.-30) .. controls +(down:2mm) .. (waitE.120);

  \end{scope}

  %% (new R.transfer ~descr:"grantE" [grantE;waitE] [idle;excl] notEx ex);
  \begin{scope}[xshift=40mm,yshift=-35mm]
    \node[transition](t){};
    \node[state,place,label={[scale=0.5,label distance=2pt]above:$waitE$},above=of t.center](i2) {};
    \node[state,place,label={[scale=0.5,label distance=0pt]above:$grantE$},left=of i2](i1) {};
    \node[state,place,label={[scale=0.5]right:$notEx$},right=of i2](i3) {};

    \node[state,place,label={[scale=0.5]below:$excl$},below=of t.center](o2) {};
    \node[state,place,label={[scale=0.5]below:$idle$},left=of o2](o1) {};
    \node[state,place,label={[scale=0.5]right:$ex$},right=of o2](o3) {};

    \draw[myedge,<-] (t.160) .. controls +(up:2mm) .. (i1.-60);
    \draw[myedge,<-] (t.90) .. controls +(up:5mm) .. (i2);
    \draw[myedge,<-,mysnake] (t.10) .. controls +(75:2mm) .. (i3);
    \draw[myedge,->] (t.-160) .. controls +(down:2mm) .. (o1.60);
    \draw[myedge,->] (t.-90) .. controls +(down:5mm) .. (o2);
    \draw[myedge,->,mysnake] (t.-10) .. controls +(-75:2mm) .. (o3);
  \end{scope}

  \draw[white,thick] (-23mm,10mm) -- +(down:55mm) (20.5mm,10mm) -- +(down:55mm) (-45mm,-17.5mm) -- +(right:100mm);

\end{tikzpicture}%
\par%
} % end \centering

Initially one token is in \emph{ex} (such that we model the exclusive
flag is $\strue$) and each token that represents a process is in
\emph{invalid}. A bad configuration is detected when the place
\emph{excl} contains 2 or more tokens.
