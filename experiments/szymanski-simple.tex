%
%
%
The full version of Szymanski's
protocol~\cite{Szymanski88,Szymanski90} is shown in
Section~\ref{section:paramsys}. We present here a compact version
taken from~\cite{APRXZ01}.

%
\begin{wrapfigure}{r}[2pt]{0.6\textwidth}
  \centering
  \begin{tikzpicture}[scale=0.8,
    mystate/.style={circle,draw=none,fill=white,thick,inner sep=1pt,scale=0.8},
    edge/.style={->,>=stealth',thin},%,shorten >=1pt,semithick},
    mylabel/.style={midway,anchor=east,scale=0.55,thick},
    myexplanation/.style={black,scale=0.6,text centered,thick},
    % mybg/.style={rounded corners,draw=none,fill=#1!10!white},
    mybg/.style={rounded corners,draw=none,shading=axis,left color=white,right color=#1!10!white},
    mybg/.default=white,
    % sepbar/.style={draw=white,line width=3pt}
    sepbar/.style={draw=black!10!white,double=white,thin,double distance=3pt}
    ]
    
    \node (n0) at (-2.5,0) [mystate] {0};
    \node (n1) at (0,3.25) [mystate] {1};
    \node (n2) at (0,2.25) [mystate] {2};
    \node (n3) at (0,1.25) [mystate] {3};
    \node (n4) at (3,0) [mystate] {4};
    \node (n5) at (0,-1.25) [mystate] {5};
    \node (n6) at (0,-2.25) [mystate] {6};
    \node (n7) at (0,-3.25) [mystate] {7};
    
    \begin{pgfonlayer}{background}
      \def\size{7}
      \path [draw,mybg=blue] (n1) ++(-1,0.5) rectangle ++(\size,-2.5) ++(0,1.25) coordinate(intention);
      \path [draw,mybg=red] (n3) ++(-1,0) rectangle ++(\size,-2.5) ++(0,1.25) coordinate(waiting);
      \path [draw,mybg=green] (n5) ++(-1,0) rectangle ++(\size,-1.5) ++(0,0.75) coordinate(cs);
      \path [draw,mybg=yellow] (n7) ++(-1,0.5) rectangle ++(\size,-1) ++(0,0.5) coordinate(exit);

      \path [draw,mybg=gray,shading=axis,left color=gray!10!white,right color=white] (n0) ++(-0.5,3.75) rectangle +(2,-7.5);

      \coordinate (doorin) at ([shift={(5.5,0)}]n3);
      \coordinate (doorout) at ([shift={(5.5,0)}]n5);

      \node at (intention) [myexplanation,left,text width=35mm,inner sep=0pt]{Standing outside the waiting room with the intention to enter, eventually blocking the doorway};
      \node at (waiting) [myexplanation,left,text width=30mm]{Waiting for the last one to enter};
      \node at (cs) [myexplanation,left,text width=30mm]{Critical Section};
      \node at (exit) [myexplanation,left,text width=5cm]{Emptying the waiting room and reopening the entry door};
      \node (nocs) at (n0) [myexplanation,right,text width=15mm]{Non critical work};

      \node (doorinText) at (doorin) [myexplanation,anchor=east,fill=white,rounded corners,inner xsep=2mm,draw=black!10!white]{Entry Door};
      \node (dooroutText) at (doorout) [myexplanation,anchor=east,fill=white,rounded corners,inner xsep=2mm,draw=black!10!white]{Exit Door};
      \draw [sepbar] (doorinText.west) ++(1pt,0) -- +(-3.65,0);
      \draw [sepbar] (dooroutText.west) ++(1pt,0) -- +(-3.8,0);
      
    \end{pgfonlayer}

    \draw [edge] (n0) -- +(0,3) to[out=90,in=90] (n1);
    \draw [edge] (n1) -- node[mylabel,right]{$\forall j\neq i : \set{0,1,2,4}$} (n2);
    \draw [edge] (n2) -- (n3);
    \draw [edge] (n3) to[out=0,in=90] node[mylabel,below left,swap,pos=0.6]{$\exists j\neq i : \set{1,2}$} (n4);
    \draw [edge] (n4) to[out=-90,in=0] node[mylabel,above left,swap,pos=0.4]{$\exists j\neq i : \set{5,6,7}$} (n5);
    \draw [edge] (n3) -- node[mylabel,right]{\begin{tabular}{c}$\forall j\neq i :$\\$\set{0,3,4,5,6,7}$\end{tabular}} (n5);
    \draw [edge] (n5) -- node[mylabel,right]{$\forall j<i : \set{0,1,2}$} (n6);
    \draw [edge] (n6) -- (n7);
    \draw [edge,<-] (n0) -- +(0,-3) to[out=-90,in=-90] node[mylabel,below right]{$\forall j>i : \set{0,1,2,5,6,7}$} (n7);

  \end{tikzpicture}
  \caption{Szymanski's protocol as a transition system. States
    correspond to program locations.}
  %\label{figure:SzymanskiTS}
\end{wrapfigure}
%
The protocol ensures exclusive access to a shared resource in a system
consisting of an unbounded number of processes organized in an array.
%
The transition diagram for the $i$th process is given in
Fig.~\ref{figure:SzymanskiTS}.
%
The protocol implements a waiting room with two doors. %
Processes intending to access the shared resource gather in the
waiting room.
%
The last one to enter closes the entry door and opens the exit door.
%
Then, processes access the resource one by one.
%
%
When all processes have left the waiting room, the exit door closes
and the entry door reopens.
%
A process may perform {\it local} transitions in which
it moves independently of the other processes,
or transitions that are guarded by {\it existential} or 
{\it universal} conditions on the states of the other processes. 
%
In the latter case, the guards may be chosen to refer only to
processes on the left or on the right of the current process.
%
Our aim is to prove correctness of the system regardless of the number
of processes and check that the protocol guarantees exclusive access
to the shared resource. A configuration $c$ is considered to be
\emph{bad} if it contains two occurrences of state $6$, ie.\
$66\subword c$. Initially, all processes are in state $0$.
