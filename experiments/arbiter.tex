\subsection{The Tree-arbiter Protocol}

The protocol supervises the access to a shared resource of a set of processes arranged in 
a tree topology.
%
The processes competing for the resource reside in the leaves.
%

A process in the protocol can be in state \emph{idle} ($i$), \emph{requesting} ($r$),  
\emph{token} ($t$) or \emph{below} ($b$).
%
All the processes are initially in state $i$.
%
A node is in state $b$ whenever it has a descendant in state $t$.
%
When a leaf is in state $r$, the request is propagated upwards
until it encounters a node which is aware of the presence of the token (i.e. a
node in state $t$ or $b$).
%
A node that has the token (in state $t$) can choose to pass it upwards or pass it downwards 
to a requesting child (node in state $r$).
%

We model the tree-arbiter protocol with a parameterized tree system $\parsys=\parsystuple$ 
where $\states=\setcomp{\state_s^n}{s\in\set{i,r,t,b}\band n\in\set{leaf,inner,root}}$ and 
$\rwrules$ is as depicted in the figure below (figure~\ref{fig:arbiter:rules}).
%
Observe that in the definition of $\states$, we use the scripts $s$ and $n$ to model respectively 
the state and the nature (leaf, inner or root) of the nodes.
%
In the definition of the rules, we will drop the script(s) whenever we mean that it is arbitrary 
(it can take any value).
%

The rules to model this protocol are as follows: 
%
2 rules to propagate the
request upwards, 
%
2 rules to propagate the token downwards, 
%
2 rules to propagate the token upwards and one rule to initiate a request from a leaf. 
%

\begin{figure}[htb]
\centering
\begin{tikzpicture}[show background rectangle]
  \begin{scope}[scale=0.5]
    \tikzstyle{every node}=[draw,rounded corners,minimum width=1cm]
    \tikzstyle{level 1}=[sibling distance=2.5cm,level distance=2cm]
    %\tikzstyle{level 2}=[sibling distance=1.8cm,level distance=1.3cm]
    \node[shift={(0cm,0cm)}] {$\state_i$/$\state_r$} child{edge from parent[draw=none]} child {node {$\state_r$}};
    \node[shift={(3cm,0cm)}] {$\state_i$/$\state_r$} child{node {$\state_r$}} child{edge from parent[draw=none]};
    \node[shift={(0cm,-2.2cm)}] {$\state_t$/$\state_b$} child {edge from parent[draw=none]} child {node {$\state_r$/$\state_t$}};
    \node[shift={(3cm,-2.2cm)}] {$\state_t$/$\state_b$} child {node {$\state_r$/$\state_t$}} child {edge from parent[draw=none]};
    \node[shift={(0cm,-4.4cm)}] {$\state_b$/$\state_t$} child{edge from parent[draw=none]} child {node {$\state_t$/$\state_i$}};
    \node[shift={(3cm,-4.4cm)}] {$\state_b$/$\state_t$} child {node {$\state_t$/$\state_i$}} child {edge from parent[draw=none]};
    \node[shift={(1.5cm,-6.6cm)}] {$\state_i^{leaf}$/$\state_r$};
  \end{scope}
  \begin{scope}[color=white]
    \draw (0cm,-1.5cm) -- +(3cm,0cm);
    \draw (0cm,-3.7cm) -- +(3cm,0cm);
    \draw (0cm,-5.9cm) -- +(3cm,0cm);
    \draw (1.5cm,0cm) -- +(0cm,-5.9cm);
  \end{scope}
\end{tikzpicture}
\caption{The rewrite rules for the tree-arbiter protocol.
%
We mention here that there are more rules in the model we have verified.
%
For example, the rule in the top-left corner is represented in the concrete model by $2$ rules, each of which 
corresponds to a particular combination of the natures of the parent and child nodes:
%
For the parent there are $2$ possibilities ($\state_i^{inner}/\state_r^{inner}$ and $\state_i^{root}/\state_r^{root}$) 
while for the child, there are $2$ ($\state_r^{inner}$ and $\state_r^{leaf}$).}
\label{fig:arbiter:rules}
\end{figure}

The set of bad constraints $\fconfs$ is represented by trees where at least two
processes (i.e. two leaves) obtain the token (i.e. in state $\state_t^{leaf}$).
%
The set of initial configurations $\init$ contains all trees where the leaf nodes
are either idle or requesting, inner nodes are idle, and the root has the token.
%
\begin{center}
  \begin{tikzpicture}[show background rectangle]
    \begin{scope}[scale=0.5]
      \tikzstyle{every node}=[draw,rounded corners,minimum width= 1cm]
      \tikzstyle{level 1}=[sibling distance=3cm,level distance=2cm]
      \node {$\state$} child {node {$\state_t^{leaf}$}} child {node {$\state_t^{leaf}$}};
    \end{scope}
    \end{tikzpicture}
\end{center}
%

