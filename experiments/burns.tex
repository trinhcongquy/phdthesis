Burns' algorithm~\cite{Burns:protocol} implements a mutual exclusion
protocol and can be modeled as a parameterized system where the
processes are arranged in a linear topology. Each process can
communicate with and distinguish its neighbors on its right or its
left.

The local state of a process ranges over state
$\set{\w[w]{1},\ldots,\w[w]{6}}$ where $\w[b]{6}$ represents the
critical section. Transitions are guarded with conditions on the
states of the neighbors, on the right, the left or both and is enabled
if the guard is not violated.

\noindent%
%\lstinputlisting[caption={Burns' mutual exclusion protocol.}, style=custom, label=figure:BurnsCode]{experiments/burns-implementation.txt}
% {\hfill
% \begin{minipage}{0.7\linewidth}
%   \lstinputlisting[style=custom]{experiments/burns-implementation.txt}
% \end{minipage}
% \hfill
% }
{\centering %
  \begin{minipage}{0.7\linewidth}
    \lstinputlisting[style=custom]{experiments/code/burns.txt}
  \end{minipage}%
  \par%
}%
%

Initially, all processes are in state $\w[i]{1}$. A bad configuration
is detected if two processes or more are in the critical section,
i.e.\ if the array contains at least two processes in state $\w[b]{6}$.
%
The transitions are depicted in the following state diagram.
%
The process $i$ is the current process, $j$ is another process and
$\w{j}$ its state.

{\noindent\centering 
  \begin{tikzpicture}[%
    node distance=3cm,
    state/.append style={scale=1.25},
    ]
    
    \node[state,state-i] (n1) at (0,0) {1};
    \node[state,state-n,right=of n1] (n2) {2};
    \node[state,state-n,right=of n2] (n3) {3};
    \node[state,state-n,below=2cm of n3] (n4) {4};
    \node[state,state-n,left=of n4] (n5) {5};
    \node[state,state-b,left=of n5] (n6) {6};
    
    \draw [->,myedge] (n1) -- (n2);
    % \draw [->,myedge] (n2) to[out=120,in=60] (n1) node[above]{$\exists j<i: \state{j}\not\in\set{1,2,3}$};
    \draw [->,myedge] (n2) .. controls +(120:0.75) and +(45:0.75) .. node[trlabel,above]{$\exists j<i: \set{4,5,6}$} (n1);
    
    \draw [->,myedge] (n2) -- node[trlabel,above]{$\forall j<i: \set{1,2,3}$} (n3);
    
    \draw [->,myedge] (n3) -- (n4);
    
    \draw [->,myedge] (n4) -- node[trlabel,below]{$\forall j<i: \set{1,2,3}$} (n5);
    
    % \draw [->,myedge] (n5) to[out=90,in=-45] (n1) node[pos=0.7,above right,sloped]{$\exists j<i: \set{4,5,6}$};
    \draw [->,myedge] (n4) to[out=135,in=-45] node[pos=0.45,trlabel,above,sloped]{$\exists j<i: \set{4,5,6}$} (n1);
    
    \draw [->,myedge] (n5) -- node[trlabel,below]{$\forall j>i: \set{1,2,3}$} (n6);
    
    \draw [->,myedge] (n6) -- (n1);
    
  \end{tikzpicture}
  \par%
}%
%\def\initstate{\tikz[baseline=(n.base)]\node[state,fill=green!20,scale=0.7](n){1};}
%\caption{A pseudocode of the $i$th process of
%  Bruns's protocol and the corresponding transition rules (in the form of a transition diagram). A state of a process is composed form
%  a program location and a value of the local variable
%  $\mathit{flag}[i]$. Since value of $\mathit{flag}[i]$ is invariant
%  at each location, states correspond to locations. Initially, all
%  processes are in state {\protect\initstate}.
%}
%\caption{Pseudocode and transition rules of Burns' protocol.}
