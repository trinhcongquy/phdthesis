%% ====================================================================
\chapterwithtoc{Case Studies}
\label{chapter:case:studies}
\label{chapter:experimentation}
%% ====================================================================

% % Suppress the sections from the Table of Content (Now only chapters)
% \addtocontents{toc}{\protect\setcounter{tocdepth}{0}}

%% =================================================
\section{Burns' mutual exclusion}
\ifnoexperiments\elseBurns' algorithm~\cite{Burns:protocol} implements a mutual exclusion
protocol and can be modeled as a parameterized system where the
processes are arranged in a linear topology. Each process can
communicate with and distinguish its neighbors on its right or its
left.

The local state of a process ranges over state
$\set{\w[w]{1},\ldots,\w[w]{6}}$ where $\w[b]{6}$ represents the
critical section. Transitions are guarded with conditions on the
states of the neighbors, on the right, the left or both and is enabled
if the guard is not violated.

\noindent%
%\lstinputlisting[caption={Burns' mutual exclusion protocol.}, style=custom, label=figure:BurnsCode]{experiments/burns-implementation.txt}
% {\hfill
% \begin{minipage}{0.7\linewidth}
%   \lstinputlisting[style=custom]{experiments/burns-implementation.txt}
% \end{minipage}
% \hfill
% }
{\centering %
  \begin{minipage}{0.7\linewidth}
    \lstinputlisting[style=custom]{experiments/code/burns.txt}
  \end{minipage}%
  \par%
}%
%

Initially, all processes are in state $\w[i]{1}$. A bad configuration
is detected if two processes or more are in the critical section,
i.e.\ if the array contains at least two processes in state $\w[b]{6}$.
%
The transitions are depicted in the following state diagram.
%
The process $i$ is the current process, $j$ is another process and
$\w{j}$ its state.

{\noindent\centering 
  \begin{tikzpicture}[%
    node distance=3cm,
    state/.append style={scale=1.25},
    ]
    
    \node[state,state-i] (n1) at (0,0) {1};
    \node[state,state-n,right=of n1] (n2) {2};
    \node[state,state-n,right=of n2] (n3) {3};
    \node[state,state-n,below=2cm of n3] (n4) {4};
    \node[state,state-n,left=of n4] (n5) {5};
    \node[state,state-b,left=of n5] (n6) {6};
    
    \draw [->,myedge] (n1) -- (n2);
    % \draw [->,myedge] (n2) to[out=120,in=60] (n1) node[above]{$\exists j<i: \state{j}\not\in\set{1,2,3}$};
    \draw [->,myedge] (n2) .. controls +(120:0.75) and +(45:0.75) .. node[trlabel,above]{$\exists j<i: \set{4,5,6}$} (n1);
    
    \draw [->,myedge] (n2) -- node[trlabel,above]{$\forall j<i: \set{1,2,3}$} (n3);
    
    \draw [->,myedge] (n3) -- (n4);
    
    \draw [->,myedge] (n4) -- node[trlabel,below]{$\forall j<i: \set{1,2,3}$} (n5);
    
    % \draw [->,myedge] (n5) to[out=90,in=-45] (n1) node[pos=0.7,above right,sloped]{$\exists j<i: \set{4,5,6}$};
    \draw [->,myedge] (n4) to[out=135,in=-45] node[pos=0.45,trlabel,above,sloped]{$\exists j<i: \set{4,5,6}$} (n1);
    
    \draw [->,myedge] (n5) -- node[trlabel,below]{$\forall j>i: \set{1,2,3}$} (n6);
    
    \draw [->,myedge] (n6) -- (n1);
    
  \end{tikzpicture}
  \par%
}%
%\def\initstate{\tikz[baseline=(n.base)]\node[state,fill=green!20,scale=0.7](n){1};}
%\caption{A pseudocode of the $i$th process of
%  Bruns's protocol and the corresponding transition rules (in the form of a transition diagram). A state of a process is composed form
%  a program location and a value of the local variable
%  $\mathit{flag}[i]$. Since value of $\mathit{flag}[i]$ is invariant
%  at each location, states correspond to locations. Initially, all
%  processes are in state {\protect\initstate}.
%}
%\caption{Pseudocode and transition rules of Burns' protocol.}
\fi

\section{Szymanski's mutual exclusion}
This example is presented in Section~\ref{section:paramsys:example}
(on page~\pageref{section:paramsys:example}).
% Simple version too? Pseudo code?

\section{Dijsktra's mutual exclusion}
\ifnoexperiments\else\label{app:dijkstra}
Dijsktra's algorithm %\cite{mutex:dijsktra}
implements a mutual exclusion protocol and can be modeled as a
parameterized system where the processes are arranged in a linear
topology. Each process can communicate with its neighbors and check
their status.

\medskip{\noindent\centering
  \begin{minipage}{0.7\linewidth}
    \lstinputlisting[style=custom]{experiments/code/dijkstra.txt}
  \end{minipage}%
  \par%
}\medskip
%
The algorithm is described in the above code listing.
%
It makes use of a pointer, i.e. a variable ranging over
process indices. We model this pointer by a local boolean variable $p$
for each process state.
%
$p$ is $\strue$ iff the pointer points to this current process. When
the pointer changes, this information must be passed onto all other
processes, which we model as a broadcast transition. Concretely, upon
pointer assignment, the current process sets its local variable $p$ to
$\strue$ and simultaneously sets $p$ to $\sfalse$ in all other
processes.

We denote the state of process as $St$ (resp. $Sf$) when the process
is in state $S$ and the pointer $p$ is $\strue$ (resp. $\sfalse$). The
state $S$ of a process ranges over $\set{\w[w]{1},\ldots,\w[w]{6}}$
where $\w[w]{6}$ represents the critical section.
%
Initially, one process is in state $\w[i]{1t}$ and all other processes
are in state $\w[i]{1f}$. A bad configuration is detected when 2 or
more processes are in the critical section, ie when their state is
either $\w[b]{6t}$ or $\w[b]{6f}$.

% \begin{figure}[h]
\noindent%
\resizebox{\linewidth}{!}{%
\begin{tikzpicture}[%
  node distance=20mm,
  state/.append style={scale=1.25},
  ]

  \node[state,state-i] (n1t) {1f};
  \node[state,state-n] (n2t) [right=of n1t] {2f};
  \node[state,state-n] (n3t) [right=of n2t] {3f};
  \node[state,state-n] (n4t) [right=of n3t] {4f};
  \node[state,state-n] (n5t) [right=of n4t] {5f};
  \node[state,state-b] (n6t) [right=of n5t] {6f};

  \node[state,state-i] (n1f) [below=of n1t] {1t};
  \node[state,state-n] (n2f) [right=of n1f] {2t};
  \node[state,state-n] (n3f) [right=of n2f] {3t};
  \node[state,state-n] (n4f) [right=of n3f] {4t};
  \node[state,state-n] (n5f) [right=of n4f] {5t};
  \node[state,state-b] (n6f) [right=of n5f] {6t};

  \draw [->,myedge] (n1t) -- (n2t);
  \draw [->,myedge] (n2t) -- node[trlabel]{\begin{tabular}{c}$\forall j\neq i :$\\$\set{[1-6]t,1f}$\end{tabular}} (n3t);
  \draw [->,myedge] (n4t) -- (n5t);
  \draw [->,myedge] (n5t) -- node[trlabel]{\begin{tabular}{c}$\forall j\neq i :$\\$\set{[1-4]\{t,f\}}$\end{tabular}} (n6t);
  \draw [->,myedge] (n5t) .. controls +(-1,1) and +(1,1) .. node[trlabel,above]{$\exists j\neq i :\set{5t,6t,5f,6f}$} (n1t);
  \draw [->,myedge] (n6t) .. controls +(0,2) and +(0,2) .. (n1t);


  \draw [->,myedge] (n3t) -- node[trlabel,sloped,above]{\emph{Broadcast}} (n4f);

  \draw [->,myedge] (n1f) -- (n2f);
  \draw [->,myedge] (n2f) to[out=-20,in=-160] (n4f);

  \draw [->,myedge] (n4f) -- (n5f);
  \draw [->,myedge] (n5f) -- node[trlabel]{\begin{tabular}{c}$\forall  j\neq i : $\\$\set{[1-4]\{t,f\}}$\end{tabular}} (n6f);
  \draw [->,myedge] (n5f) .. controls +(-1,-1) and +(1,-1) .. node[trlabel,below]{$\exists j\neq i : \set{5t,6t,5f,6f}$} (n1f);

  \draw [->,myedge] (n6f) .. controls +(0,-2) and +(0,-2) .. (n1f);

  \foreach \x in {1,2,3,4,5,6}{ \draw [<->,myedge,draw=gray] (n\x t) -- node[rotate=45,scale=0.5,midway]{\color{gray}Global change} (n\x f); }     

  %% Uncomment to see the limits of the bounding box
  % \fill[red] (current bounding box.north west) circle (2pt);
  % \fill[red] (current bounding box.south east) circle (2pt);
\end{tikzpicture}
} % End resise
% \caption{The transitions of Dijkstra's protocol}
% \label{figure:dijkstra:transitions}
% \end{figure}
\fi

\newpage
\section{Gribomont-Zenner's mutual exclusion}
\ifnoexperiments\elseThis algorithm could be seen as a version of Szymanski's
algorithm~\ref{figure:szymanski:implementation}, with transitions that
are finer-grained in the sense that tests and assignments are split
over different atomic transitions.
%
In this model, the local state of a process ranges over
state~$\set{\w[w]{1},\ldots,\w[w]{13}}$ where \w[i]{1} is the initial
state and \w[b]{12} represents the critical section.  Configurations
not satisfying mutual exclusion are those where at least two processes
are at state~\w[b]{12}.

{\noindent%
  %\centering
  \resizebox{\linewidth}{!}{%
  \begin{tikzpicture}[%
        node distance=25mm,
        state/.append style={minimum width=6mm}%
  ]     

  \node[state,state-i] (n1) {1};
  \node[state,state-n] (n2) [left=10mm of n1] {2};
  \node[state,state-n] (n3) [left=10mm of n2] {3};
  \node[state,state-n] (n4) [below=10mm of n3] {4};
  \node[state,state-n] (n5) [below=10mm of n4] {5};
  \node[state,state-n] (n6) [right=of n5] {6};
  \node[state,state-n] (n7) [right=of n6] {7};
  \node[state,state-n] (n8) [right=of n7] {8};
  \node[state,state-n] (n9) [right=of n8] {9};
  \node[state,state-n] (n10) [above=10mm of n9] {10};
  \node[state,state-n] (n11) [above=10mm of n10] {11};
  \node[state,state-b] (n12) [left=of n11] {12};
  \node[state,state-n] (n13) [left=6mm of n12] {13};

   \draw [->,myedge] (n1) -- (n2);
   \draw [->,myedge] (n2) -- (n3);
   \draw [->,myedge] (n3) -- node[trlabel,right]{$\forall j\neq i : \set{1,2,3,4,7,8}$} (n4);
   \draw [->,myedge] (n4) -- (n5);
   \draw [->,myedge] (n5) -- node[trlabel,above]{$\exists j\neq i : \set{3,4,10-13}$} (n6);
   \draw [->,myedge] (n6) -- (n7);
   \draw [->,myedge] (n7) -- node[trlabel,above]{$\exists j\neq i : \set{10-13}$} (n8);
   \draw [->,myedge] (n8) -- (n9);
   \draw [->,myedge] (n9) -- (n10);
   \draw [->,myedge] (n10) -- node[trlabel,left]{$\forall j\neq i: \set{1-4,10-13}$} (n11);
   \draw [->,myedge] (n11) -- node[trlabel,above]{$\forall j<i: \set{1-4,7,8}$} (n12);

   \draw [->,myedge] (n12) -- (n13);
   \draw [->,myedge] (n13) -- (n1);

   %\draw [->,myedge] (n5) to[out=-30,in=-150] node[mylabel,above,pos=0.5]{$\forall j\neq i : \set{1,2,5-9}$} (n9);
   \draw [->,myedge] (n5) ..controls +(-45:10mm) and +(-135:10mm).. (n9) node[trlabel,below,pos=0.5]{$\forall j\neq i : \set{1,2,5-9}$};

\end{tikzpicture}%
}% End resize
%\par%
}
%\def\initstate{\tikz[baseline=(n.base)]\node[state,fill=green!20,scale=0.7](n){1};}
%\caption{A pseudocode of the $i$th process of
%  Gribomont-Zenner's protocol and the corresponding transition rules (in the form of a transition diagram). 
%  Initially, all processes are in state {\protect\initstate}.
%}
%\caption{Transition rules of Gribomont-Zenner's protocol.}
%
\fi

\section{Parosh's mutual exclusion}
\ifnoexperiments\elseThis protocol ensures mutual exclusion between processes.
Each process has five local states %
\w[w]{0},%
\w[w]{1},%
\w[w]{2},%
\w[w]{3},%
\w[w]{4} %
and is initially in state\,\w[i]{0}. %

\noindent%
\begin{wrapfigure}{r}{0.5\textwidth}
  \centering
  \begin{tikzpicture}[%scale=0.8,
    % mystate/.style={circle,draw=none,fill=white,thick,inner sep=1pt,scale=0.8},
    edge/.style={->,>=stealth',thin},%,shorten >=1pt,semithick},
    trlabel/.style={midway,anchor=east,scale=0.9,thick},
    guard/.style={state,state-n,scale=0.8,thin,circle}
    ]
    
    \node[state,state-i] (n0) at (0,0)     {0};
    \node[state,state-n] (n1) at (1,1.2)   {1};
    \node[state,state-n] (n2) at (2.4,1.2) {2};
    \node[state,state-n] (n3) at (3.4,0)   {3};
    \node[state,state-b] (n4) at (1.7,0)   {4};
    
    \draw [edge,->] (n0) to[out=90,in=-180] node[trlabel,above left,pos=0.25](l01){$\forall$} (n1);
    \draw [edge,->] (n1) -- (n2);
    \draw [edge,->] (n2) to[out=0,in=90] node[trlabel,above right,pos=0.7](l23){$\forall_L$} (n3);
    \draw [edge,->] (n3) -- node[trlabel,above,pos=0.6](l34){$\exists$} (n4);
    \draw [edge,->] (n4) -- (n0);
    \draw [edge,->] (n3)  ..controls +(-135:8mm) and +(-45:8mm).. (n0);

    %% 0 -> 1 forall 0,1,4
    \node[guard,state-i,right=-0.5mm of l01]{0};
    \node[guard,right=1.5mm of l01]{1};
    \node[guard,state-b,right=3.5mm of l01]{4};
    %% 2 -> 3 forall_left 0
    \node[guard,state-i,right=0mm of l23]{0};
    %% 3 -> 4 exists 2
    \node[guard,right=-0.5mm of l34]{2};
  \end{tikzpicture}
  % \caption{State diagram (per process).}
  % \label{figure:parosh}
\end{wrapfigure}
%
A process in the critical section is at state\,\w[b]{4}. %
The set of bad configurations contains exactly configurations with at
least two occurrences of state\,\w[b]{4}.
%
Processes move from state~\w[w]{0} to~\w[w]{1}, and
then~\w[w]{2}.
% 
Once the first process is in state~\w[w]{2}, it ``closes the door'' on
the processes which are still in~\w[w]{0}. They can no longer leave
state~\w[w]{0} until the door is opened again (when no process is in
state~\w[w]{2} or~\w[w]{3}). %
%
Moreover, a process is allowed to cross from state~\w[w]{3}
to~\w[w]{4} only if there is at least one process still in
state~\w[w]{2} (i.e., the door is still closed on the processes in
state~\w[w]{0}). %
%
This prevents a process to first reach state~\w[w]{4} along with
a process to its left starting to move from~\w[w]{0} all the way
to state~\w[w]{4} (thus violating mutual exclusion). %
From the set of processes which have left state~\w[w]{0} (and
which are now in state~\w[w]{1} or~\w[w]{2}), the leftmost
process has the highest priority and it is encoded in the global
condition: a process may move from \w[w]{2} to~\w[w]{3}
only if all processes on its left are in state~\w[w]{0}.
%

We have shown in Paper~\ref{paper:SAS14}, using the view abstraction
method from Section~\ref{chapter:view:abstraction}, that this
protocol is not safe in the case of non-atomic global transitions.
\fi

\section{Bakery mutual exclusion}
\ifnoexperiments\else%
\begin{wrapfigure}{r}[0pt]{0.3\linewidth}
\hfill%
\begin{tikzpicture}[%
  state/.append style={minimum width=6mm}%
  ]

  \node[state,state-i] (n1) at (0,0) {1};
  \node[state,state-n] (n2) at (1,1) {2};
  \node[state,state-b] (n3) at (2,0) {3};

   \draw [->,myedge] (n1) to[out=90,in=180] node[trlabel,above,sloped]{$\forall j<i : \set{1}$} (n2);
   \draw [->,myedge] (n2) to[out=0,in=90] node[trlabel,above,sloped]{$\forall j>i : \set{1}$} (n3);
   \draw [->,myedge] (n3) -- (n1);

\end{tikzpicture}
%\def\initstate{\tikz[baseline=(n.base)]\node[state,fill=green!20,scale=0.7](n){1};}
%\caption{A pseudocode of the $i$th process of
%  Bakery's protocol and the corresponding transition rules (in the form of a transition diagram). 
%  Initially, all processes are in state {\protect\initstate}.
%}
%\caption{Transition rules of Bakery's protocol.}
\end{wrapfigure}
%
This case study describes a simplified version of the original Bakery
algorithm~\cite{Lamport:Bakery}.
%
In this version~\cite{Marcus:thesis}, processes have states that range
over~$\set{\w[w]{1},\w[w]{2},\w[w]{3}}$, where \w[i]{1} is the initial
state.
%
A process gets a ticket with a value strictly higher than the ticket
value of any process in the queue (transition
\w[w]{1}$\rightarrow$\w[w]{2}).
%
A process accesses the critical section if it has a ticket with the
lowest value among the existing tickets (transition
\w[w]{2}$\rightarrow$\w[w]{3}).
%
Finally, a process leaves the critical section, freeing its ticket
(transition \w[w]{3}$\rightarrow$\w[w]{1}).
%
Mutual exclusion violation corresponds to configurations where more
than one process is in state~\w[b]{3}.
\fi


%% =================================================
\section{MOSI Cache Coherence Protocol}
\ifnoexperiments\elseThe MOSI protocol is an extension of the basic MSI cache coherency
protocol. It is a snoop-based protocol. It adds the state Owned~\s{O},
which indicates that the current processor owns this block, and will
service requests from other processors for the block. This also
reduces the amount of write-back data upon cache eviction.

The protocol can be modeled as a parameterized system where the
processes are arranged in a multiset. Each process can communicate
with its neighbors by broadcasting a message on the connecting bus.

\begin{wrapfigure}{r}[2pt]{0.35\linewidth}
  \newcommand{\wrong}{\raisebox{-0.3pt}{\includegraphics[height=1em]{img/skull.png}}}
  \newcommand{\valid}{{\color{green}\bf$\checkmark$}}
  \newcommand{\cache}[1]{\bf\small\emph{#1}}
  \hfill%
  \begin{tabular}{c|cccc}
                & {\cache{M}} & {\cache{O}} & {\cache{S}} & {\cache{I}} \\\hline
    {\cache{M}} & {\wrong}    & {\wrong}    & {\wrong}    & {\valid}       \\
    {\cache{O}} & {\wrong}    & {\wrong}    & {\valid}    & {\valid}       \\
    {\cache{S}} & {\wrong}    & {\valid}    & {\valid}    & {\valid}       \\
    {\cache{I}} & {\valid}    & {\valid}    & {\valid}    & {\valid}       \\
  \end{tabular}
\end{wrapfigure}
%
The state of a cache line can be \emph{Modified}~\s{M},
\emph{Owned}~\s{O}, \emph{Shared}~\s{S}, and \emph{Invalid}~\s{I}.
%
The broadcast communication is depicted using the two following
automata.
%
Initially, all cache lines are in state \emph{Invalid}~\s{I}. The
permitted states of any given pair of cache lines is given in the
table beside.

The first automaton represents the action taken when the given process
issues the broadcast message and/or when it manipulates the cache
line.
%
The cache can be written (\emph{CPUwrite}), caused by a store miss, it
can be read (\emph{CPUread}) caused by a load miss and finally, it can
be replaced (\emph{CPUrepl}).
%
The message sent on the bus vary depending on the state of the current
cache line. The active process can send a \emph{read-to-share}
(\emph{BUSrts}) request, a \emph{read-to-write} (\emph{BUSrtw})
request, a \emph{write-back} (\emph{BUSwb}) request (in case of cache
eviction) or an \emph{invalidate} (\emph{BUSinv}) request. We label
each edge of the automaton with the action
%
\raisebox{-3pt}{\begin{tikzpicture}
  \begin{pgfinterruptboundingbox}
    \path node[message](n){\emph{action}\nodepart{second}\emph{message}} [clip] (n.west) rectangle (n.north east);
  \end{pgfinterruptboundingbox}
  \path[use as bounding box,baseline=(n.one base)] (n.west) rectangle (n.north east);
\end{tikzpicture}}
%
issued by the current process and with the message %
\raisebox{-5pt}{\begin{tikzpicture}
  \begin{pgfinterruptboundingbox}
    \path node[message](n){\emph{action}\nodepart{second}\emph{message}} [clip] (n.south west) rectangle (n.east);
  \end{pgfinterruptboundingbox}
  \path[use as bounding box,baseline=(n.second base)] (n.south west) rectangle (n.east);
\end{tikzpicture}}
%
it sent on the bus. We use
%
\raisebox{-5pt}{\begin{tikzpicture}
  \begin{pgfinterruptboundingbox}
    \path node[message](n){\emph{action}\nodepart{second}$-$} [clip] (n.south west) rectangle (n.east);
  \end{pgfinterruptboundingbox}
  \path[use as bounding box] (n.south west) rectangle (n.east);
\end{tikzpicture}}
%
when no message is sent on the bus.

\smallskip%
{\noindent\centering
  \begin{tikzpicture}[node distance=3cm,
    state/.append style={state-n,inner sep=2pt,scale=1.5},
    message/.append style={scale=0.5}
    ]

    \node[state] (m)	              {M};
    \node[state] (o) [right=4cm of m] {O};
    \node[state] (s) [below=of m]     {S};
    \node[state] (i) [below=of o]     {I};

    \draw [->,myedge] (m) .. controls +(-0.5,1) and +(-1,0.5) ..
    node[message,anchor=north east,pos=0.7]{\emph{CPUread}\nodepart{second}$-$}
    node[message,anchor=south east,pos=0.3]{\emph{CPUwrite}\nodepart{second}$-$} (m);
    \draw [->,myedge] (m) to[out=-30,in=120] node[message,anchor=south,sloped]{\emph{CPUrepl}\nodepart{second}\emph{BUSwb}} (i);
    \draw [->,myedge] (o) -- node[message,above]{\emph{CPUwrite}\nodepart{second}\emph{BUSinv}} (m);
    \draw [->,myedge] (o) -- node[message,right]{\emph{CPUrepl}\nodepart{second}\emph{BUSwb}} (i);
    \draw [->,myedge] (o) .. controls +(0.5,1) and +(1,0.5) .. node[message,anchor=south west]{\emph{CPUread}\nodepart{second}$-$} (o);
    \draw [->,myedge] (s) to[out=-30,in=-150] node[message,anchor=north]{\emph{CPUrepl}\nodepart{second}$-$} (i);
    \draw [->,myedge] (s) .. controls +(-1,-0.5) and +(-0.5,-1) .. node[message,anchor=north east]{\emph{CPUread}\nodepart{second}$-$} (s);
    \draw [->,myedge] (s) -- node[message,left]{\emph{CPUwrite}\nodepart{second}\emph{BUSinv}} (m);
    \draw [->,myedge] (i) -- node[message,anchor=south]{\emph{CPUread}\nodepart{second}\emph{BUSrts}} (s);
    \draw [->,myedge] (i) to[out=150,in=-60] node[message,anchor=north,sloped]{\emph{CPUwrite}\nodepart{second}\emph{BUSrtw}} (m);
    
  \end{tikzpicture}
  \par%
}

\smallskip%
The second automaton represents how other processes react upon
reception of a message %
\raisebox{-5pt}{\begin{tikzpicture}
  \begin{pgfinterruptboundingbox}
    \path node[message](n){\emph{action}\nodepart{second}\emph{message}} [clip] (n.south west) rectangle (n.east);
  \end{pgfinterruptboundingbox}
  \path[use as bounding box,baseline=(n.second base)] (n.south west) rectangle (n.east);
\end{tikzpicture}}.

\smallskip%
{\noindent\centering
  \begin{tikzpicture}[node distance=3cm,
    state/.append style={state-n,inner sep=2pt,scale=1.5},
    message/.append style={scale=0.5,rectangle split parts=1,fill=pink!10!white,rectangle split part fill={yellow!10!white}}
    ]

    \node[state] (m)	              {M};
    \node[state] (o) [right=4cm of m] {O};
    \node[state] (s) [below=of m]     {S};
    \node[state] (i) [below=of o]     {I};
    
    \draw [->,myedge] (m) -- node[message,anchor=south]{\emph{BUSrts} (+Data)} (o);
    \draw [->,myedge] (m) -- node[message,sloped,above]{\emph{BUSrtw} (+Data)} (i);
    \draw [->,myedge] (o) -- node[message,anchor=west] {\ensuremath{\begin{tabular}{l} \emph{BUSrtw} (+Data) \\ \emph{BUSinv}\end{tabular}}} (i);
    \draw [->,myedge] (s) -- node[message,anchor=south east,pos=0.6]{\ensuremath{\begin{tabular}{l}\emph{BUSrtw}\\\emph{BUSinv}\end{tabular}}} (i);
    \draw [->,myedge] (s) .. controls +(-1,-0.5) and +(-0.5,-1) .. node[message,anchor=east]{\ensuremath{\begin{tabular}{l}\emph{BUSrts}\\\emph{BUSwb}\end{tabular}}} (s);
    \draw [->,myedge] (i) .. controls +(1,-0.5) and +(0.5,-1) .. node[message,anchor=south west]{\ensuremath{\begin{tabular}{l}\emph{BUSrts}\\\emph{BUSrtw}\\\emph{BUSinv}\\\emph{BUSwb}\end{tabular}}} (i);
    \draw [->,myedge] (o) .. controls +(0.5,1) and +(1,0.5) .. node[message,anchor=south west]{\emph{BUSrts} (+Data)} (o);
  \end{tikzpicture}
  \par%
}
\fi
\section{German Cache Coherence Protocol}
\ifnoexperiments\elseThe directory-based cache protocol consists of a central server $H$
(for \emph{Home}), and a set of client caches $C_1,\ldots,C_n$. Each
client $C_i$ communicates with the server through a set of three
message channels namely $ch_1^i,ch_2^i,ch_3^i$ in a star
topology.

A client $C_i$ can request to share a given cache and sends a
\emph{ReqShared} or \emph{ReqExclusive} through its $ch_1^i$. The
\emph{Home} node will reply in $ch_2^i$ with an \emph{Invalidate}, a
\emph{GrantShared}, or a \emph{GrantExclusive} message. The client
aknowledges the reception by sending a \emph{InvAck} message in
$ch_3^i$.

A cache line can be in state \emph{Invalid}, \emph{Shared} or
\emph{Exclusive}. If a cache line is in state \emph{Invalid} in the
cache $C_i$, the client $i$ does not have access to that cache line.
If a client has been granted the access, by \emph{Home}, possibly
along with other clients, the cache line is in state \emph{Shared}.
\emph{Home} can also grant the access exclusively to a client, if
which case the cache line is in state \emph{Exclusive}.

\begin{wrapfigure}{r}[2pt]{0.45\textwidth}
  \hfill
  \begin{tikzpicture}[auto,>=stealth']
    \node[scale=0.7,ellipse,ball color=blue!50!white](home){Home};
    \foreach \i/\t/\a in {1/$C_1$/60, 2/$C_2$/120, 3/$C_3$/180, 4/$C_4$/240, 5/$\ldots$/-80, 6/$\ldots$/-40, 7/$C_6$/360}{
      \node(c\i)[scale=0.8,circle,ball color=green!50!white] at +(\a:23mm){\t};
      \begin{pgfonlayer}{background}
        \draw[->] (home.\a) .. controls +(\a:1cm) and +(\a:-1cm) .. (c\i) coordinate[midway](ch2\i);
        \draw[<-] ([rotate=5]home.\a) .. controls +([rotate=5]\a:1cm) and +(\a:-1cm) .. ([rotate=5]c\i) coordinate[pos=0.2](ch1\i);
        \draw[<-] ([rotate=-5]home.\a) .. controls +([rotate=-5]\a:1cm) and +(\a:-1cm) .. ([rotate=-5]c\i) coordinate[pos=0.8](ch3\i);
      \end{pgfonlayer}
      \node[scale=0.4,anchor=east,inner sep=0pt, fill=white!90!gray] at (ch11){$ch_1$};
      \node[scale=0.4,anchor=center,inner sep=0pt, fill=white!90!gray] at (ch21){$ch_2$};
      \node[scale=0.4,anchor=north west,inner sep=0pt, fill=white!90!gray] at (ch31){$ch_3$};
    }
  \end{tikzpicture}
\end{wrapfigure}
%
Initially, all channels are empty and all cache lines are in state
\emph{Invalid}. A bad configuration is detected when two (or more)
processes have exclusive access to a given cache line, or if one
accesses the cache line exclusively whilst the others still have a
shared access.

We model each client in a parametrized system with multiset topology.
The actions of $H$ are represented in each process and its (bounded)
local variables are modeled as shared variables. The model
follows~\cite{PRZ-tacas01}. We assume the channels to be of length one
and therefore can represent them by a local variable (for each
client).  In addition to channels, the central controller manipulates
five data:
\begin{description}[leftmargin=6em,style=nextline,align=right,labelsep=\parindent]
\item[\color{blue} excl] a flag to remember whether the exclusive access has been granted.% (modeled with a shared boolean variable).
\item[\color{blue} ctl] a pointer to the client that sent the request being served.% (modeled with a local boolean variable),
\item[\color{blue} sh\_list] a list of the processes having an access, either shared or exclusive, to the cache~line. % (modeled with a local boolean variable), and
\item[\color{blue} inv\_list] a list of  processes which have to be invalidated in order to serve the current request.% (modeled also with a local boolean variable). 
\item[\color{blue} cmd] a message read in some buffer.% (modeled as shared variable). 
\end{description}%
% We model a variable of the server as follows: Each client has a local
% copy of it. When the server performs an update to that shared
% variable, a broadcast communication is issued: all clients update
% their local copy simultaneously.

A client $i$ may perform one of the following actions:
\begin{description}[leftmargin=6em,style=nextline,align=right,labelsep=\parindent]
\item[\color{orange} $c_1$] If in state invalid and $ch_1^i$ is empty, the client $i$ sends a request to $H$
  for a shared access via its $ch_1$.
\item[\color{orange} $c_2$] If in state invalid or shared while $ch_1^i$ is empty, the client $i$ sends 
  a request to $H$ for an exclusive access via its $ch_1$.
\item[\color{orange} $c_3$] If the client $i$ is granted a shared access via $ch_2^i$,
  it consumes the message from the channel and update the state of the
  cache line to \emph{Shared}. 
\item[\color{orange} $c_4$] If the client $i$ is granted an exclusive access via $ch_2^i$,
  it consumes the message from the channel and update the state of the
  cache line to \emph{Exclusive}.
\item[\color{orange} $c_5$] If the client $i$ receives an invalidation message through
  $ch_2^i$ and $ch_3^i$ is empty, it changes the state of the cache
  line to \emph{Invalid}, empties $ch_2^i$ and sends an invalidation
  acknowledgment to $H$ via $ch_3^i$.
\end{description}

Depending on the content of the channels and the values of the shared
variables, $H$ may perform one of the following actions.
\begin{description}[leftmargin=6em,style=nextline,align=right,labelsep=\parindent]
\item[\color{orange} $h_1$] If $H$ is idle (ie $cmd$ is empty) and receives a request via
  some $ch_1$, it consumes the received request from $ch_1$ into
  $cmd$, selects the sender to be the current client and copies the
  content of the sharer list to the invalidation list.
\item[\color{orange} $h_2$] In case some $ch_2$ is empty, the current command is a
  shared request and the exclusive access has not been granted, then
  $H$ grants a shared access to the current client via $ch_2$ and adds
  the client to the shared list and returns idle.
\item[{\color{orange} $h_3$},{\color{orange}~$h_4$}] If the current command is either a shared request
  (while the exclusive flag is set) or an exclusive request, some
  $ch_2$ is empty, then $H$ sends an invalidation message every
  process through $ch_2$ and removes these processes from the
  invalidation list.
\item[\color{orange} $h_5$] In case the current command is a request for either a
  shared or an exclusive access and $H$ receives an invalidation
  acknowledgment from a client via $ch_3$, then $H$ removes a client
  from the sharer list, resets the exclusive flag and empties $ch_3$.
\end{description}
%
Using this model, the safety properties we checked are: (i) no two
clients are simultaneously granted an exclusive access, and (ii) no
client in state shared coexists with a client in state exclusive.
%
We described the model, the variables and transitions in plain
text. The complete description and pseudocode can be found
in~\cite{PRZ-tacas01}. 

%% ================================================================
\subsection*{Modeled as a Petri Net.}
A simplified version of the protocol has been modeled as a Petri Net
as in~\cite{Raskin:experiments:German}. It uses 12 places:
\emph{idle,serveS,serveE,grantS,} and \emph{grantE} model which
request is being served, \emph{ex,notEx} to model the exclusive flag,
\emph{waitS,waitE} to model the response from $H$, and finally
\emph{shared,excl,invalid} to model the cache line states.


{\noindent\centering %
\begin{tikzpicture}[show background rectangle,
  node distance=6mm,
  label distance=2pt,
  ]

  %% (new R.apply ~descr:"reqS" [idle;invalid] [serveS;waitS]);
  \begin{scope}[xshift=-40mm]
    \node[transition](t){};
    \node[state,place,label={[scale=0.5]above:$idle$},above=of t.west](i1){};
    \node[state,place,label={[scale=0.5]above:$inv$},above=of t.east](i2) {};
    \node[state,place,label={[scale=0.5]below:$serveS$},below=of t.west](o1) {};
    \node[state,place,label={[scale=0.5]below:$waitS$},below=of t.east](o2) {};
    \draw[myedge,<-] (t.150) .. controls +(up:5mm) .. (i1);
    \draw[myedge,<-] (t.30) .. controls +(up:5mm) .. (i2);
    \draw[myedge,->] (t.-150) .. controls +(down:5mm) .. (o1);
    \draw[myedge,->] (t.-30) .. controls +(down:5mm) .. (o2);
  \end{scope}

  %% (new R.apply ~descr:"invalid" [serveS;ex;excl] [notEx;grantS;invalid]);
  \begin{scope}
    \node[transition](t){};
    \node[state,place,label={[scale=0.5]above:$shared$},above=of t.center](i2) {};
    \node[state,place,label={[scale=0.5]above:$serveE$},left=of i2](i1) {};
    \node[state,place,label={[scale=0.5]above:$excl$},right=of i2](i3) {};
    \node[state,place,label={[scale=0.5]below:$grantE$},below=of t.west](o1) {};
    \node[state,place,label={[scale=0.5]below:$inv$},below=of t.east](o2) {};
    \draw[myedge,<-] (t.150) .. controls +(up:5mm) .. (i1);
    \draw[myedge,<-,decorate] (t.30) .. controls +(up:5mm) .. (i2);
    \draw[myedge,<-,mysnake] (t.15) .. controls +(75:2mm) .. (i3);
    \draw[myedge,->] (t.-150) .. controls +(down:5mm) .. (o1);
    \draw[myedge,->,mysnake] (t.-30) .. controls +(-75:5mm) .. (o2);
  \end{scope}

  %% (new R.apply ~descr:"nonex" [serveS;notEx] [grantS;notEx]);
  \begin{scope}[xshift=40mm]
    \node[transition](t){};
    \node[state,place,label={[scale=0.5]above:$serveS$},above=of t.west](i1) {};
    \node[state,place,label={[scale=0.5]right:$notEx$},right=of t](notex) {};
    \node[state,place,label={[scale=0.5]below:$grantS$},below=of t.west](o1) {};
    \draw[myedge,<-] (t.150) .. controls +(up:5mm) .. (i1);
    \draw[myedge,<-] (t.30) .. controls +(up:5mm) and +(120:5mm) .. (notex);
    \draw[myedge,->] (t.-150) .. controls +(down:5mm) .. (o1);
    \draw[myedge,->] (t.-30) .. controls +(down:5mm) and +(-120:5mm) .. (notex);
  \end{scope}

  %% (new R.apply ~descr:"grantS" [grantS;waitS] [idle;shared]);
  \begin{scope}[xshift=-40mm,yshift=-35mm]
    \node[transition](t){};
    \node[state,place,label={[scale=0.5]above:$grantS$},above=of t.west](i1) {};
    \node[state,place,label={[scale=0.5]above:$waitS$},above=of t.east](i2) {};
    \node[state,place,label={[scale=0.5]below:$idle$},below=of t.west](o1) {};
    \node[state,place,label={[scale=0.5]below:$shared$},below=of t.east](o2) {};
    \draw[myedge,<-] (t.150) .. controls +(up:5mm) .. (i1);
    \draw[myedge,<-] (t.30) .. controls +(up:5mm) .. (i2);
    \draw[myedge,->] (t.-150) .. controls +(down:5mm) .. (o1);
    \draw[myedge,->] (t.-30) .. controls +(down:5mm) .. (o2);
  \end{scope}

  %% (new R.apply ~descr:"reqE" [idle;invalid] [serveE;waitE]);
  %% (new R.apply ~descr:"reqE2" [idle;shared] [serveE;waitE]);
  \begin{scope}[yshift=-26mm]% magic number
    \node[state,place,label={[scale=0.5]above:$idle$}](idle){};
    \node[state,place,label={[scale=0.5]above:$shared$},left=of idle](shared) {};
    \node[state,place,label={[scale=0.5]above:$inv$},right=of idle](inv) {};

    \node[transition,minimum width=7mm,below=of shared,anchor=west](t1){};
    \node[transition,minimum width=7mm,below=of inv,anchor=east](t2){};

    \node[state,place,label={[scale=0.5]below:$serveE$},below=of t1](serveE) {};
    \node[state,place,label={[scale=0.5]below:$waitE$},below=of t2](waitE) {};

    \draw[myedge,<-] (t1.150) .. controls +(up:2mm) .. (shared);
    \draw[myedge,<-] (t1.30) .. controls +(up:2mm) .. (idle);
    \draw[myedge,->] (t1.-150) .. controls +(down:2mm) .. (serveE.60);
    \draw[myedge,->] (t1.-30) .. controls +(down:2mm) .. (waitE.120);

    \draw[myedge,<-] (t2.150) .. controls +(up:2mm) .. (idle);
    \draw[myedge,<-] (t2.30) .. controls +(up:2mm) .. (inv);
    \draw[myedge,->] (t2.-150) .. controls +(down:2mm) .. (serveE.60);
    \draw[myedge,->] (t2.-30) .. controls +(down:2mm) .. (waitE.120);

  \end{scope}

  %% (new R.transfer ~descr:"grantE" [grantE;waitE] [idle;excl] notEx ex);
  \begin{scope}[xshift=40mm,yshift=-35mm]
    \node[transition](t){};
    \node[state,place,label={[scale=0.5,label distance=2pt]above:$waitE$},above=of t.center](i2) {};
    \node[state,place,label={[scale=0.5,label distance=0pt]above:$grantE$},left=of i2](i1) {};
    \node[state,place,label={[scale=0.5]right:$notEx$},right=of i2](i3) {};

    \node[state,place,label={[scale=0.5]below:$excl$},below=of t.center](o2) {};
    \node[state,place,label={[scale=0.5]below:$idle$},left=of o2](o1) {};
    \node[state,place,label={[scale=0.5]right:$ex$},right=of o2](o3) {};

    \draw[myedge,<-] (t.160) .. controls +(up:2mm) .. (i1.-60);
    \draw[myedge,<-] (t.90) .. controls +(up:5mm) .. (i2);
    \draw[myedge,<-,mysnake] (t.10) .. controls +(75:2mm) .. (i3);
    \draw[myedge,->] (t.-160) .. controls +(down:2mm) .. (o1.60);
    \draw[myedge,->] (t.-90) .. controls +(down:5mm) .. (o2);
    \draw[myedge,->,mysnake] (t.-10) .. controls +(-75:2mm) .. (o3);
  \end{scope}

  \draw[white,thick] (-23mm,10mm) -- +(down:55mm) (20.5mm,10mm) -- +(down:55mm) (-45mm,-17.5mm) -- +(right:100mm);

\end{tikzpicture}%
\par%
} % end \centering

Initially one token is in \emph{ex} (such that we model the exclusive
flag is $\strue$) and each token that represents a process is in
\emph{invalid}. A bad configuration is detected when the place
\emph{excl} contains 2 or more tokens.
\fi

%% =================================================
\section{Tree Protocols}
\noindent%
\begin{wrapfigure}{r}{0.25\linewidth}    
  \hfill%
  \resizebox{\linewidth}{!}{%
  \begin{tikzpicture}[show background rectangle,
    level 1/.style={sibling distance=15mm,level distance=10mm},
    every node/.style={state,state-n,inner sep=2pt,solid}
    ]
    
    \node {$p_1/p_2$}
    child { node{$l_1/l_2$} edge from parent }
    child { node{$r_1/r_2$} edge from parent };
  \end{tikzpicture}
  } % End resizebox
\end{wrapfigure}
% 
In this section, we present the tree protocols that have been used in
benchmarks in the papers of this thesis.
%
For a tree transition, if the tree pattern $p_1,l_1,r_1$ is found in
the tree, the rule is applied and the nodes change their local state
to $p_2,l_2,r_2$ respectively.
% 
We represent the rule in a compact form on the right.

\subsection{Token Protocol}
\ifnoexperiments\else\index{Token Protocol}

The protocol operates on binary trees to transmit a token from the
root to the leaves. A node can be labeled as having the token~\s{t},
or not having the token~\s{n}.
%
Initially, the token is in the root.
%
As an example:

{\noindent\centering
  \begin{tikzpicture}[show background rectangle,
    level 1/.style={sibling distance=10mm,level distance=12mm},
    level 2/.style={sibling distance=5mm,level distance=12mm},
    every node/.style={state,state-n,inner sep=2pt,solid},
    ]
    \node(t1) {t}
    child {node {n}
      child {node {n}}
      child {node {n}}}
    child {node {n}
      child {node {n}}
      child {node {n}}
    };
    \node(t2)[xshift=40mm] {n}
    child {node {t}
      child {node {n}}
      child {node {n}}}
    child {node {u}
      child {node {n}}
      child {node {n}}
    };
    \node(t3)[xshift=80mm] {n}
    child {node {n}
      child {node {n}}
      child {node {t}}}
    child {node {n}
      child {node {n}}
      child {node {n}}
    };

    %% Small step arrows
    \foreach \f/\t in {1/2,2/3}{
      \path (t\f) -- coordinate[midway](t\f\t) (t\t);
      \draw[*->] (t\f\t) ++(-3mm,0) .. controls +(3mm,1mm) .. +(6mm,0);
    }
  \end{tikzpicture}
  \par%
}

The set of bad constraints \Bad\ is represented by trees where at
least two nodes contain the token.

{\noindent\centering
  \begin{tikzpicture}[show background rectangle,
    sibling distance=10mm,level distance=12mm,
    every node/.style={state,state-n,inner sep=2pt,solid},
    ]
    \node {t} child {node {t}} child[missing];
  \end{tikzpicture}
  \begin{tikzpicture}[show background rectangle,
    sibling distance=10mm,level distance=12mm,
    every node/.style={state,state-n,inner sep=2pt,solid},
    ]
    \node {t} child[missing] child {node {t}};
  \end{tikzpicture}
  \begin{tikzpicture}[show background rectangle,
    sibling distance=10mm,level distance=12mm,
    every node/.style={state,state-n,inner sep=2pt,solid},
    ]
    \node {*} child {node{t}} child {node {t}};
  \end{tikzpicture}
  \par%
}


\fi
\subsection{Two-way Token Protocol}
\ifnoexperiments\else\index{Two-way Token}

This protocol is a generalization of the token protocol by allowing
the token to both move upwards and downwards.
%
Initially, the token is in the root.
%
The set of bad constraints \Bad\ is represented by trees where at
least two nodes contain the token, similarly to the simple token
protocol.
%
The rules for the propagation of the token are:

{\noindent\centering
  \begin{tikzpicture}[show background rectangle, sibling distance=10mm,level distance=12mm,
    every node/.style={state,state-n,inner sep=2pt,solid} ]
    \node {t / n} child {node {n / t}} child[missing];
  \end{tikzpicture}
  \begin{tikzpicture}[show background rectangle, sibling distance=10mm,level distance=12mm,
    every node/.style={state,state-n,inner sep=2pt,solid} ]
    \node {t / n} child[missing] child {node {n / t}};
  \end{tikzpicture}
  \begin{tikzpicture}[show background rectangle, sibling distance=10mm,level distance=12mm,
    every node/.style={state,state-n,inner sep=2pt,solid} ]
    \node {n / t} child {node {t / n}} child[missing];
  \end{tikzpicture}
  \begin{tikzpicture}[show background rectangle, sibling distance=10mm,level distance=12mm,
    every node/.style={state,state-n,inner sep=2pt,solid} ]
    \node {n / t} child[missing] child {node {t / n}};
  \end{tikzpicture}
  \par%
}

\newpage
As an example:

{\noindent\centering
  \begin{tikzpicture}[show background rectangle,
    level 1/.style={sibling distance=10mm,level distance=12mm},
    level 2/.style={sibling distance=5mm,level distance=12mm},
    every node/.style={state,state-n,inner sep=2pt,solid},
    ]
    \node(t1)[red] {t}
    child {node {n}
      child {node {n}}
      child {node {n}}}
    child {node {n}
      child {node {n}}
      child {node {n}}
    };
    \node(t2)[xshift=30mm] {n}
    child {node[red] {t}
      child {node {n}}
      child {node {n}}}
    child {node {u}
      child {node {n}}
      child {node {n}}
    };
    \node(t3)[xshift=60mm] {n}
    child {node {n}
      child {node {n}}
      child {node[red] {t}}}
    child {node {n}
      child {node {n}}
      child {node {n}}
    };
    \node(t4)[xshift=90mm] {n}
    child {node[red] {t}
      child {node {n}}
      child {node {n}}}
    child {node {u}
      child {node {n}}
      child {node {n}}
    };
    \node(t5)[xshift=120mm] {n}
    child {node {n}
      child[red] {node {t}}
      child {node {n}}}
    child {node {n}
      child {node {n}}
      child {node {n}}
    };

    %% Small step arrows
    \foreach \f/\t in {1/2,2/3,3/4,4/5}{
      \path (t\f) -- coordinate[midway](t\f\t) (t\t);
      \draw[*->] (t\f\t) ++(-3mm,0) .. controls +(3mm,1mm) .. +(6mm,0);
    }
  \end{tikzpicture}
  \par%
}

\fi
\subsection{The Tree Percolate Protocol}
\ifnoexperiments\else\index{Percolate}

The protocol~\cite{KMMPS2001} operates on binary trees of processes
and evaluates the disjunction of the values in the leaves up to the
root.
%
The states are $\set{\s{0}, \s{1}, \s{u}}$.
%
Initially, the leaves contain either~\s{0} or~\s{1} and all
other nodes are labeled as undefined~\s{u}.
%
A still undefined inner node will be labeled as~\s{1} if at least one
of its children contains~\s{1}, and~\s{0} otherwise.
%
As an example:

{\noindent\centering%
  \begin{tikzpicture}[show background rectangle,
    level 1/.style={sibling distance=10mm,level distance=12mm},
    level 2/.style={sibling distance=5mm,level distance=12mm},
    every node/.style={state,state-n,inner sep=2pt,solid},
    ]
    \node(t1) {$u$}
    child {node {$u$}
      child {node {$0$}}
      child {node {$0$}}}
    child {node {$u$}
      child {node {$1$}}
      child {node {$1$}}
    };
    \node(t2)[xshift=40mm] {$u$}
    child {node {$0$}
      child {node {$0$}}
      child {node {$0$}}}
    child {node {$1$}
      child {node {$1$}}
      child {node {$1$}}
    };
    \node(t3)[xshift=80mm] {$1$}
    child {node {$0$}
      child {node {$0$}}
      child {node {$0$}}}
    child {node {$1$}
      child {node {$1$}}
      child {node {$1$}}
    };

    %% Small step arrows
    \foreach \f/\t in {1/2,2/3}{ \path (t\f) -- coordinate[midway](t\f\t) (t\t); }
    \foreach \x in {t12,t23}{ \draw[*->] (\x) ++(-3mm,0) .. controls +(3mm,1mm) .. +(6mm,0); }
  \end{tikzpicture}
  \par%
}

The rules are depicted as follows.

{\noindent\centering%
  \begin{tikzpicture}[show background rectangle,
    level 1/.style={sibling distance=15mm,level distance=10mm},
    every node/.style={state,state-n,inner sep=2pt}
    ]
    
    \node[solid](p_1){$u/0$}
    child { node{$0/0$} edge from parent }
    child { node{$0/0$} edge from parent };
    
    \node[right=25mm of p_1](p_2){$u/1$}
    child { node{$1/1$} edge from parent }
    child { node{$0/0$} edge from parent };
    
    \node[right=25mm of p_2](p_3){$u/1$}
    child { node{$0/0$} edge from parent }
    child { node{$1/1$} edge from parent };
    
    \node[right=25mm of p_3](p_4){$u/1$}
    child { node{$1/1$} edge from parent }
    child { node{$1/1$} edge from parent };
    
  \end{tikzpicture}
  \par%
}


The set of bad constraints \Bad\ is represented by trees where \s{0}
gets propagated upwards while there is \s{1} below.

{\noindent\centering%
  \begin{tikzpicture}[show background rectangle,
    level 1/.style={sibling distance=15mm,level distance=10mm},
    every node/.style={state,state-n,inner sep=2pt}
    ]
    \node {$0$} child {node {$1$}} child {edge from parent[draw=none]};
  \end{tikzpicture}
  \begin{tikzpicture}[show background rectangle,
    level 1/.style={sibling distance=15mm,level distance=10mm},
    every node/.style={state,state-n,inner sep=2pt}
    ]
    \node {$0$} child {edge from parent[draw=none]} child {node {$1$}};
  \end{tikzpicture}
  \par%
}
\fi
\subsection{The Leader Election Protocol}
\ifnoexperiments\else\index{Leader Election}\index{Leader}

The protocol operates on binary trees to elect a leader among
processes which reside in the leaves and which are candidates.
%
A leaf process can be labeled as candidate~\s{c} or
non-candidate~\s{n}.
%
An inner node is initially labeled as undefined~\s{u} and will
be labeled as candidate if at least one of its children is candidate,
and non-candidate otherwise.

In a first phase, the information that a node is candidate or not
travels up to reach the root.
%
In a second phase, the decision~\s{el}, initially in the root,
travels down from candidate parent to candidate child.
%
(If several children are candidates, the parent chooses one
undeterministicly).
%
Once the decision reaches a leaf, this leaf process is elected as
leader.
%

%\noindent%
\begin{wrapfigure}{r}{0.25\linewidth}
  \hfill%
  \resizebox{\linewidth}{!}{%
  \begin{tikzpicture}[show background rectangle,
    level 1/.style={sibling distance=15mm,level distance=10mm},
    every node/.style={state,state-n,inner sep=2pt, solid}
    ]
    \node {*} child { node{el} edge from parent } child { node{el} edge from parent };
  \end{tikzpicture}
  } % end resizebox
\end{wrapfigure}
% 
The set of initial constraints \Inits\ is represented by trees where
leaves are either candidates or non-candidates, inner nodes are
labeled undefined and the root is labeled~\s{el}.
%
The set of bad constraints \Bad\ is represented by trees where at
least two nodes in different branches are elected (on the right).
%
The rules for the upwards propagation of candidate information are:

{\noindent\centering%
  \begin{tikzpicture}[show background rectangle,
    level 1/.style={sibling distance=10mm,level distance=12mm},
    level 2/.style={sibling distance=5mm,level distance=12mm},
    every node/.style={state,state-n,inner sep=2pt,solid},
    ]
    \node                {u / c} child {node {c}} child {node {n}}; 
    \node[shift={(2,0)}] {u / c} child {node {n}} child {node {c}};
    \node[shift={(4,0)}] {u / c} child {node {c}} child {node {c}};
    \node[shift={(6,0)}] {u / n} child {node {n}} child {node {n}};
  \end{tikzpicture}
  \par%
}

\noindent%
The rules for the downwards propagation of election decision are:

{\noindent\centering%
  \begin{tikzpicture}[show background rectangle,
    level 1/.style={sibling distance=10mm,level distance=12mm},
    every node/.style={state,state-n,inner sep=2pt,solid},
    ]
    \node {el} child {node {c / el}} child {edge from parent[draw=none]};
  \end{tikzpicture}
  %
  \begin{tikzpicture}[show background rectangle,
    level 1/.style={sibling distance=10mm,level distance=12mm},
    every node/.style={state,state-n,inner sep=2pt,solid},
    ]
    \node {el} child {edge from parent[draw=none]} child {node {c / el}};
  \end{tikzpicture}
  \par%
}

Below follows a small scenario:

{\noindent\centering%
  \resizebox{\linewidth}{!}{%
  \begin{tikzpicture}[show background rectangle,
    level 1/.style={sibling distance=10mm,level distance=12mm},
    level 2/.style={sibling distance=5mm,level distance=12mm},
    every node/.style={state,state-n,inner sep=2pt,solid},
    ]

    \node(t1) {el}
    child {node {u} child {node {n} edge from parent } child {node {c} edge from parent }}
    child {node {u} child {node {n}} child {node {n}}};
    \node(t2)[xshift=30mm] {el}
    child {node[red] {c} child[red] {node {n}} child[red] {node {c}}}
    child {node {u} child {node {n}} child {node {n}}};
    \node(t3)[xshift=60mm] {el}
    child {node {c} child {node {n}} child {node {c}}}
    child {node[red] {n} child[red] {node {n}} child[red] {node {n}}};
    \node(t4)[xshift=90mm,red] {el}
    child[red] {node {el} child[black] {node {n}} child[black] {node {c}}}
    child {node {n} child {node {n}} child {node {n}}};
    \node(t5)[xshift=120mm] {el}
    child {node[red] {el} child {node {n}} child[red] {node {el}}}
    child {node {n} child {node {n}} child {node {n}}};

    %% Small step arrows
    \foreach \f/\t in {1/2,2/3,3/4,4/5}{ \path (t\f) -- coordinate[midway](t\f\t) (t\t); }
    \foreach \x in {t12,t23,t34,t45}{ \draw[*->] (\x) ++(-3mm,0) .. controls +(3mm,1mm) .. +(6mm,0); }

  \end{tikzpicture}
  } % end resizebox
  \par%
}
\fi
\subsection{The Tree Arbiter Protocol}
\ifnoexperiments\else\index{Arbiter}\index{Tree-Arbiter Protocol}

The protocol supervises the access to a shared resource of a set of
processes arranged in a tree topology.
%
The processes competing for the resource reside in the leaves.
%
A process in the protocol can be in state \emph{idle}~\s{i},
\emph{requesting}~\s{r}, \emph{token}~\s{t} or
\emph{below}~\s{b}.
%
All the processes are initially in state~\s{i}.
%
A node is in state~\s{b} whenever it has a descendant in
state~\s{t}.
%
When a leaf is in state~\s{r}, the request is propagated upwards
until it encounters a node which is aware of the presence of the token
(i.e. a node in state~\s{t} or~\s{b}).
%
A node that has the token (in state~\s{t}) can choose to pass it
upwards or pass it downwards to a requesting child (node in
state~\s{r}).
%

We model the tree-arbiter protocol with a parameterized tree system
$\parsys=(\locs,\rules)$ where
$\locs=\setcomp{\s{\ensuremath{s_n}}}{s\in\set{\s{i},\s{r},\s{t},\s{b}}\wedge
  n\in\set{leaf,inner,root}}$
and $\rules$ is as depicted below, in Figure~\ref{figure:tree:arbiter:rules}.
%
Observe that in the definition of $\locs$, we use the subscript~$n$ to
model the nature (leaf, inner or root) of the nodes.
%
In the definition of the rules, we will drop the script whenever we
mean that it is arbitrary (it can take any value).
%

The rules to model this protocol are as follows: 
%
2 rules to propagate the request upwards,
%
2 rules to propagate the token downwards, 
%
2 rules to propagate the token upwards and one rule to initiate a request from a leaf. 
%

\begin{figure}[hb]
  \centering
  \begin{tikzpicture}[show background rectangle,
    level 1/.style={sibling distance=15mm,level distance=10mm},
    every node/.style={state,state-n,inner sep=2pt}
    ]

    \node(t1)                    {i/r} child[missing] child {node {r}};
    \node(t2)[right=20mm of t1]  {i/r} child {node {r}} child[missing];
    \node(t3)[right= 5mm of t2]  {t/b} child[missing] child {node {r/t}};
    \node(t4)[right=20mm of t3]  {t/b} child {node {r/t}} child[missing];
    \node(t5)[right= 5mm of t4]  {b/t} child[missing] child {node {t/i}};
    \node(t6)[right=20mm of t5]  {b/t} child {node {t/i}} child[missing];
    \coordinate[right=8mm of t6](t7); % cheating
    \node[below=5mm of t7]  {i$_{leaf}$/r};

    %\foreach \f/\t in {1/2,2/3,3/4,4/5,5/6}{
    \foreach \f/\t in {2/3,4/5}{
      \path (t\f) -- coordinate[midway](t\f\t) (t\t);
      \draw[color=white] (t\f\t) -- +(0,-10mm); % vertical
    }
    \draw[color=white] (t6.east) ++(2mm,0) -- +(0,-10mm); % vertical t67
  \end{tikzpicture}
  \label{figure:tree:arbiter:rules}
  \caption{The rewrite rules for the tree-arbiter protocol.
    % 
    Notice that there are more rules in the model:
    % 
    For example, the first rule on the left is represented in the concrete model by $2$ rules, each of which 
    corresponds to a particular combination of the natures of the parent and child nodes:
    % 
    For the parent there are $2$ possibilities
    (\s{$i_{inner}/r_{inner}$} and \s{$i_{root}/r_{root}$}) while for
    the child, there are $2$ (\s{$r_{inner}$} and \s{$r_{leaf}$}).}
\end{figure}

\begin{wrapfigure}{r}{0.3\linewidth}
  \resizebox{\linewidth}{!}{%
  \begin{tikzpicture}[show background rectangle,
    level 1/.style={sibling distance=15mm,level distance=10mm},
    every node/.style={state,state-n,inner sep=2pt}
    ]
    \node {*} child {node {$t_{leaf}$}} child {node {$t_{leaf}$}};
  \end{tikzpicture}
  } % end resizebox
\end{wrapfigure}
%
The set of initial configurations \Inits\ contains all trees where
the leaf nodes are either idle or requesting, inner nodes are idle,
and the root has the token.
%
The set of bad constraints \Bad\ is represented by trees where at
least two processes obtained the token (i.e. two leaves in
state~\s{$t_{leaf}$}, see the figure on the right).
\fi
\subsection{The IEEE 1394 Tree Identification Protocol}
\ifnoexperiments\else\index{Firewire}

The 1394 High Performance serial bus~\cite{ieee:1394:firewire} is used
to transport digitized video and audio signals within a network of
multimedia systems and devices.
% 

The tree identification protocol is used in one of the phases
implementing the IEEE 1394 protocol.
%
More precisely, it is run after a bus reset in the network and leads
to the election of a unique leader node.

In this section, we consider a version working on tree topologies.
%
Furthermore, we assume that (i) each inner node is connected to $3$
neighbors, (ii) the root is connected to $2$ neighbors, and (ii)
communication is atomic.

Initially, all nodes are in state \emph{undefined}~\s{u}. 
%
We identify two steps in the protocol depending on the number~$n$ of
neighbors which are still in state~\s{u}.
%
If $n>1$, the node waits for (``be my parent'') requests from its neighbors. 
%
If $n=1$, the node sends a request to the remaining neighbor in state~\s{u}. 
%
We can observe that the leaf nodes are the first to communicate with
their neighbors.

Formally, we derive a parameterized tree system model $\parsys$ as follows.
%
We define the set of states by
$\locs=\setcomp{\s{\ensuremath{s_n}}}{s\in\set{\s{u},\s{c},\s{\ensuremath{\ell}}}\wedge
  n\in\set{leaf,inner,root}}$
where the script $n$ describe the nature of the node.
%
In the definition of the state ($s$), the letters \s{u}, \s{c} and
\s{$\ell$} stand respectively for \emph{undefined}, \emph{child} and
\emph{leader}.
%
In a similar manner to the previous protocol, we drop the script
whenever we mean that it can take any value (see caption of
Figure~\ref{figure:tree:arbiter:rules}).
% 

The rewrite rules $\rules$ are described as follows.
%
\begin{itemize}
\item The leaves initiate the communications:

{\noindent\centering
  \begin{tikzpicture}[show background rectangle,
    level 1/.style={sibling distance=15mm,level distance=10mm},
    every node/.style={state,state-n,inner sep=2pt}
    ]
    \node {u} child{node{u$_{leaf}$ / c}} child[missing];
  \end{tikzpicture}
  % 
  \begin{tikzpicture}[show background rectangle,
    level 1/.style={sibling distance=15mm,level distance=10mm},
    every node/.style={state,state-n,inner sep=2pt}
    ]
    \node {u} child[missing] child{node{u$_{leaf}$ / c}};
  \end{tikzpicture}
  \par%
}

\item The inner nodes become children or wait for requests:

{\noindent\centering
  \begin{tikzpicture}[show background rectangle,
    grow cyclic,
    level 1/.style={level distance=1cm,sibling angle=120},
    every node/.style={state,state-n,inner sep=2pt}
    ]
    \node {u / c} child {node {u}} child {node {c}} child {node {c}};
  \end{tikzpicture}
  % 
  \begin{tikzpicture}[show background rectangle,
    grow cyclic,
    level 1/.style={level distance=1cm,sibling angle=120},
    every node/.style={state,state-n,inner sep=2pt}
    ]
    \node {u / c} child {node {c}} child {node {u}} child {node {c}};
  \end{tikzpicture}
  % 
  \begin{tikzpicture}[show background rectangle,
    grow cyclic,
    level 1/.style={level distance=1cm,sibling angle=120},
    every node/.style={state,state-n,inner sep=2pt}
    ]
    \node {u / c} child {node {c}} child {node {c}} child {node {u}};
  \end{tikzpicture}
  \par%
}

\item The leader is chosen:

{\noindent\centering
  \begin{tikzpicture}[show background rectangle,
    grow cyclic,
    level 1/.style={level distance=1cm,sibling angle=120},
    every node/.style={state,state-n,inner sep=2pt}
    ]
    \node {u / $\ell$} child {node {c}} child {node {c}} child {node {c}};
  \end{tikzpicture}
  % 
  \begin{tikzpicture}[show background rectangle,
    grow cyclic,
    level 1/.style={level distance=1cm,sibling angle=120},
    every node/.style={state,state-n,inner sep=2pt}
    ]
    \node {c} child{node {u$_{leaf}$ / $\ell$}} child[missing] child[missing];
  \end{tikzpicture}
  % 
  \begin{tikzpicture}[show background rectangle,
    grow cyclic,
    level 1/.style={level distance=1cm,sibling angle=120},
    every node/.style={state,state-n,inner sep=2pt}
    ]
    \begin{scope}[rotate=60]
      \node {c} child{node {u$_{leaf}$ / $\ell$}} child[missing] child[missing];
    \end{scope}
  \end{tikzpicture}
  \par%
}
\end{itemize}
%

The set of initial configurations \Inits\ is represented by trees where all nodes
are in state undefined, and the set of bad constraints \Bad\ is represented by 
trees where at least 2 leaders are elected.

{\noindent\centering
  \begin{tikzpicture}[show background rectangle, level distance=1cm, every node/.style={state,state-n,inner sep=2pt}]
    \node {$\ell$} child {node {$\ell$}} child[missing];
  \end{tikzpicture}
  % 
  \begin{tikzpicture}[show background rectangle, level distance=1cm, every node/.style={state,state-n,inner sep=2pt}]
    \node {$\ell$} child[missing] child {node {$\ell$}};
  \end{tikzpicture}
  % 
  \begin{tikzpicture}[show background rectangle, level distance=1cm, every node/.style={state,state-n,inner sep=2pt}]
    \node {*} child {node {$\ell$}} child {node {$\ell$}};
  \end{tikzpicture}
  \par%
}
\fi

%% =================================================
\section{Agreement protocol on a Ring}
\ifnoexperiments\else%
Interacting peers are organized in a circular pipeline and are given a
number. The protocol in place is to ensure that every participant in
the ring knows which number in the maximum among the values of the
ring members. We model this protocol as an instance of the framework
on rings. Each participant in the ring can communicate with its
adjacent neighbors. We assume the ring oriented and a member can send
messages to its (immediate) ``successor'' and receive messages from
its ``predecessor''. The reception is usually \emph{blocking} while
the sending is not. However, we will model this communication as a
rendez-vous.

\noindent%
\begin{wrapfigure}{r}[0pt]{0.25\linewidth}
  \hfill%
  \begin{tikzpicture}
    [scale=0.8,every node/.style={scale=0.8,circle,inner sep=1pt,fill=red!20}, show background rectangle] 

    \draw (0,0) .. controls +(up:15mm) and +(up:15mm) .. (3,0)
    \foreach \n/\pos in {1/0.2,2/0.5,3/0.9}{node[state,pos=\pos](p\n){P$_{\n}$}};
    \draw (0,0) .. controls +(down:15mm) and +(down:15mm) .. (3,0)
    \foreach \n/\pos in {0/0.1,5/0.5,4/0.8}{node[state,pos=\pos](p\n){P$_{\n}$}};
    \draw[->,myedge,thick] ([xshift=1mm,yshift=1mm]p2.east) to[out=-5,in=120] ([xshift=1mm,yshift=1mm]p3.north west);
  \end{tikzpicture}
\end{wrapfigure}
%
Initially, all process are in a \emph{dormant} state. One of the
process will wake up first. This is the one which initializes the
protocol (denoted in the pseudocode as $P_0$). Every process sends the
max between its own value and the value its received from its
predecessor (the biggest value it has seen so far). Note that the
protocol doesn't not terminate when $P_0$ receives the biggest value
in the ring. It must indeed communicate this value to the others. Each
will receive it from its predecessor and only pass it along to its
successor (after recording it). The protocol terminates when the
predecessor of $P_0$ receives the biggest value and avoids the
re-sending. A bad configuration is detected if one of the participant
is in its final state (\s{6} or \s{17}) but has not seen the
biggest value go by. %
We depict the protocol using the following pseudocode. The channel
between $P_{i-1}$ and $P_i$ is called {\tt values[n]}. A message in
the channel will contain the number the sender sent.

\bigskip
\lstinputlisting[style=custom]{experiments/code/agreement.txt}
\fi

%% =================================================
\newpage
\section{Critical section guarded by a lock}
% This example is presented in
% Section~\ref{section:monotonic:abstraction:applications:pn} (on
% page~\pageref{section:monotonic:abstraction:applications:pn}).
\ifnoexperiments\else\begin{wrapfigure}{r}[2pt]{0.55\textwidth}
  \hfill
  \begin{tikzpicture}[scale=0.7,
    show background rectangle,
    %
    shared/.style={draw=green,fill=green!30!gray,thick},
    dottedrectangle/.style={rounded corners,draw,dotted,inner sep=1pt},
    localVars/.style={ellipse,minimum height=30mm,minimum width=12mm, shade, top color=yellow, bottom color=white},
    globalVars/.style={ellipse, shade, top color=blue!50!red, bottom color=white},
    % 
    every edge/.style={draw,shorten >=1pt,>=stealth',semithick},
    every label/.append style={scale=0.7},
    every place/.append style={scale=0.7},
    ]
    
    \node(cheating) at(-2,0){};
    
    \node[state,place,label=right:$init$](in)  at (4,0) {?};
    \node[state,place](lockOut) at (0,0) {};
    \node[state,place](temp1) at (0,-2) {};
    \node[state,place](readOut) at (0,-4) {};
    \node[state,place](temp2) at (0,-6) {};
    \node[state,place](writeOut) at (0,-8) {};
    \node[state,place,label=right:$end$](unlockOut) at (4,-8) {};
    
    \node[transition,rotate=90,label={[scale=0.9]above left:$lock$}](tlock) at (2,0){} edge [pre] (in) edge [post] (lockOut);
    \node[transition](treadIn) at (0,-1){} edge [pre] (lockOut) edge [post] (temp1);
    \node[transition](treadOut) at (0,-3){} edge [pre] (temp1) edge [post] (readOut);
    \node[transition](twriteIn) at (0,-5){} edge [pre] (readOut) edge [post] (temp2);
    \node[transition](twriteOut) at (0,-7){} edge [pre] (temp2) edge [post] (writeOut);
    \node[transition,rotate=90,label={[scale=0.9]below right:$unlock$}](tunlock) at (2,-8){} edge [pre] (writeOut) edge [post] (unlockOut);
    
    \node[state,place,shared,label=right:$L$,tokens=1] (L)  at (6,-4) {};
    \node[state,place,shared,label=right:$\LD{counter}$] (Rc)  at (2,-3.5) {};
    \node[state,place,shared,label=right:$\ST{counter}$] (Wc)  at (2,-4.5) {};
    
    \draw[pre,myedge] (tlock.-165) to[out=0,in=90] (L);
    \draw[post,myedge] (tunlock.-15) to[out=0,in=-90] (L);
    
    \draw[post,myedge] (treadIn.-15) to[out=-90,in=120] (Rc.120);
    \draw[pre,myedge] (treadOut.15) to[out=90,in=150] (Rc.150);
    
    \draw[post,myedge] (twriteIn.-15) to[out=-90,in=-150] (Wc.-150);
    \draw[pre,myedge] (twriteOut.15) to[out=90,in=-120] (Wc.-120);
    
    \begin{pgfonlayer}{my background}
      \node[dottedrectangle,fit=(Rc) (Wc)] (r){};
      \node[dottedrectangle,fit=(treadIn) (treadOut)] (read){};
      \node[dottedrectangle,fit=(twriteIn) (twriteOut)] (write){};
    \end{pgfonlayer}
    
  \end{tikzpicture}    
\end{wrapfigure}
%
We can model the critical section problem by read and write access to
a resource shared by multiple processes. There is no particular
topology so we will model it as an instance of a parametrized system
with multisets. More precisely, as a Petri Net.
%
As the petri net of a concurrent program, described at that level of
granularity, can quickly grow in size, we choose a short example: The
processes repeatedly grab the lock, increment a \emph{counter} and
release the lock. A bad configuration is detected when two or more
processes are accessing the shared variable simultaneously while one
is writing.

The shared variable $counter$ is associated with two places,
$\LD{counter}$ and $\ST{counter}$. The tokens of a place in the petri
net represent the count of process in a given state, or the available
resources. %
A process places a token in $\LD{counter}$ (resp. $\ST{counter}$) if
it is currently accessing the variable $counter$ for reading
(resp. writing). We model read and write accesses to shared variables
with two transitions, denoted by the dotted rectangle in the following
figure.
%
There is a place $L$ associated with a lock. Intuitively, if $L$
contains a token, the lock is free, otherwise it is busy. This ensures
that only one process can hold the lock at a time. 
%
Note that $L$ is a global variable and that we omit the input places
used to balance out the number of tokens in the net.
%

Initially, the lock is free and the processes are in the initial state
\emph{init}. The petri contains then one token in $L$ and the others
in \emph{init}. 
%
A bad situation is detected when the petri net contains two or more
tokens in $\ST{counter}$ or when there is one (or more) token in
$\LD{counter}$ and one (or more) token in $\ST{counter}$.
\fi

\section{Light Control}
\ifnoexperiments\elseThis algorithm implements a simple solution to the light system of an
office room. An arbitrary number of people may get in or out of the
room.
%
Initially, the light is off and everyone is outside the room.
%
The first person to enter the room turns the light on and the last
person to exit the room turns it off.
%

%
A bad configuration is detected when the light is on, but there is
noone in the room, or when the light is off while there are still
people in the room.

We model this algorithm with a Petri Net with Inhibitor arcs.  There
are 2 places associated with the \emph{on} and \emph{off} status of
the light. %
And there are 2 places associated with the fact that a person is
inside or outside the room. %
An extra place \emph{exiting} is used to check the person who wishes
to exit is the last one in the room. %
Initially, there is a token in the \emph{off} place, and all the other
tokens in the outside place. %
For readibility, we represent the transitions piecewise as follows.

%   let is_bad m = m.(bad) > 0
%     (* (m.(fst inside) + m.(fst exiting) > 0 && m.(fst off) > 0) || ( m.(fst inside) + m.(fst exiting) = 0 && m.(fst on) > 0) *) 

{\noindent\centering
\resizebox{\linewidth}{!}{%
\begin{tikzpicture}[show background rectangle,
  trlabel/.style={node distance=1mm,scale=0.8},
  caption/.style={mylabel,anchor=north},
  every place/.append style={scale=0.7}
  ]
  %% Going in
%     R.T ( [outside;off], [inside;on], "[outside;off] -> [inside;on]");
%     R.T ( [outside;on], [inside;on], "[outside;on] -> [inside;on]");

  \begin{scope}%[xshift=-4cm]

    \node[state,place](on) at (2,0) {};
    \node[state,place](off)  at (2,-2) {};
    \node[state,place](in)  at (0,0) {};
    \node[state,place](out)  at (0,-2) {};

    \node[trlabel,right=of on]{on};
    \node[trlabel,right=of off]{off};
    \node[trlabel,left=of in]{in};
    \node[trlabel,left=of out]{out};
    
    \node[transition](t1) at (1.7,-1){};
    \draw[myedge,<-] (t1.-165) to[out=-90,in=45] (out);
    \draw[myedge,<-] (t1.-15) to[out=-90,in=135] (off);
    \draw[myedge,->] (t1.165) to[out=90,in=-45] (in);
    \draw[myedge,->] (t1.15) to[out=90,in=-135] (on);

    \node[transition](t2) at (0.3,-1){};
    \draw[myedge,<-] (t2.-165) to[out=-90,in=135] (out);
    \draw[myedge,<-] (t2.-15) ..controls +(-65:10mm) and +(-160:13mm).. (on);
    \draw[myedge,->] (t2.165) to[out=90,in=-135] (in);
    \draw[myedge,->] (t2.15) to[out=90,in=180] (on);

    \node[caption] at (1,-2.4) {Going in};
  \end{scope}

  %% Going out
  \begin{scope}[shift={(4,0)}]

    \node[state,place](on) at (2,0) {};
    \node[state,place](off)  at (2,-2) {};
    \node[state,place](in)  at (0,0) {};
    \node[state,place](out)  at (0,-2) {};
    \node[state,place](exiting) at (1,-1) {};

    \node[trlabel,right=of on]{on};
    \node[trlabel,right=of off]{off};
    \node[trlabel,left=of in]{in};
    \node[trlabel,left=of out]{out};
    \node[trlabel,above=0mm of exiting]{exiting};

%     R.Inhibitor ( [inside], [exiting], [fst exiting], "[inside] empty[exiting] -> [exiting]");
    \node[transition](t1) at (-0.1,-1){};
    \draw[myedge,<-] (t1.165) to[out=90,in=-90] (in);
    \draw[myedge,*-] (t1.15) ..controls +(up:10mm) and +(left:6mm).. (exiting);
    \draw[myedge,->] (t1) ..controls +(-60:8mm) and +(down:8mm).. (exiting);

%     R.Inhibitor ( [exiting;on], [outside;off], [fst inside], "[exiting;on] empty[inside] -> [outside;off]");
    \node[transition](t2) at (2.1,-1){};
    \draw[myedge,<-] (t2.15) to[out=90,in=-90] (on);
    \draw[myedge,<-] (t2.165) ..controls +(up:10mm) and +(right:6mm).. (exiting);
    \draw[myedge,*-] (t2.90) to[out=90,in=0] (in);
    \draw[myedge,->] (t2.-165) to[out=-90,in=0] (out);
    \draw[myedge,->] (t2.-15) to[out=-90,in=90] (off);


    \node[caption] at (1,-2.4) {Going out};
  \end{scope}

  %% Still going out
  \begin{scope}[shift={(8,0)}]

    \node[state,place](on) at (2,0) {};
    \node[state,place](off)  at (2,-2) {};
    \node[state,place](in)  at (0,0) {};
    \node[state,place](out)  at (0,-2) {};
    \node[state,place](exiting) at (1,-1) {};

    \node[trlabel,right=of on]{on};
    \node[trlabel,right=of off]{off};
    \node[trlabel,left=of in]{in};
    \node[trlabel,left=of out]{out};
    \node[trlabel,above=0mm of exiting]{exiting};

%     R.T ( [exiting;inside], [outside;inside], "[exiting;inside] -> [outside;inside]");
    \node[transition](t1) at (-0.1,-1){};
    \draw[myedge,<-] (t1.165) to[out=90,in=-90] (in);
    \draw[myedge,<-] (t1.15) ..controls +(up:10mm) and +(left:6mm).. (exiting);
    \draw[myedge,->] (t1.-165) ..controls +(-135:10mm) and +(-135:8mm).. (in);
    \draw[myedge,->] (t1.-15) to[out=-90,in=45] (out);

    %% Bad cases
    \node[state,place,draw=black,fill=black!80!white](bad) at (3,-1) {};
    \node[trlabel,below=0mm of bad]{Bad};
%     R.T ( [inside;off], [(bad,2)], "[inside;off] -> [bad;bad]");
    \node[transition,rotate=90,minimum width=6mm,draw=black](t2) at (2.3,-1.5){};
    \draw[myedge,<-] (t2.35) ..controls +(left:5mm) and +(right:15mm).. (in.-10);
    \draw[myedge,<-] (t2.135) to[out=180,in=135] (off);
    \draw[myedge,->] (t2.-65) to[out=0,in=170] (bad);
    \draw[myedge,->] (t2.-125) to[out=0,in=-170] (bad);

%     R.Inhibitor ( [on], [(bad,1)], [fst inside; fst exiting], "[on] empty[inside;exiting] -> [bad]");
    \node[transition,rotate=90,minimum width=6mm,draw=black](t3) at (2.1,-0.7){};
    \draw[myedge,<-] (t3.15) ..controls +(left:5mm) and +(left:5mm).. (on);
    \draw[myedge,*-] (t3.90) ..controls +(left:5mm) and +(right:15mm).. (in.10);
    \draw[myedge,*-] (t3.165) to[out=180,in=0] (exiting);
    \draw[myedge,->] (t3.-90) to[out=0,in=135] (bad);


    \node[caption] at (1,-2.4) {Not the last out \& Bad};
  \end{scope}

  \draw[white,thick] (3,0) -- +(0,-2.5);
  \draw[white,thick] (7,0) -- +(0,-2.5);
\end{tikzpicture}
}% end \resizebox
\par} % end \centering
\fi

\section{Priority Allocator}
\ifnoexperiments\elseA ferry transports one car at a time from one side of the river to the
other side.
%
At the departure, there are 2 queues of cars: One with high priority,
one without priority.
%
If there are cars in the high priority lane, the ferry must load them
first. If not, it can load a car from the lane with no priority, not
even with a first-come-first-served basis.

Initially, the cars must choose their respective lane, but are not
allowed to board the ferry, which is not loaded.
%
A bad configuration is detected when the ferry has loaded a car from
the no-priority lane while there are still cars in the high-priority
lane.


We model this situation with a Petri Net with Inhibitor arcs. %
There are 2 places associated with the \emph{high} and \emph{low}
priority lanes. %
And there are 2 places associated with the fact that a car has been
granted access on the ferry. %
There is an initialization phase that models how the car get
attributed a priority. %

{\noindent\centering
\resizebox{\linewidth}{!}{%
\begin{tikzpicture}[show background rectangle,
  label distance=2pt,
  caption/.style={mylabel,anchor=north},
  every place/.append style={scale=0.7}
  ]

  \begin{scope}

    \node[state,place,label={[scale=0.6]left:$Start$}](start) at (0,20mm) {};
    \node[state,place,label={[scale=0.6]above:$Phase1$}](phase1) at (0,10mm) {};
    \node[state,place,label={[scale=0.6]above:$Phase2$}](phase2) at (2,15mm) {};

    \node[state,place,label={[scale=0.6]right:$High$}](high) at (10mm,-10mm) {};
    \node[state,place,label={[scale=0.6]left:$Low$}](low)  at (-10mm,-10mm) {};

    \node[transition](t1) at (-10mm,0){};
    \draw[myedge,<-] (t1.165) to[out=90,in=-135] (start);
    \draw[myedge,<-] (t1.15) to[out=90,in=180] (phase1);
    \draw[myedge,->] (t1.-15) ..controls +(down:10mm) and +(-100:3mm).. (phase1);
    \draw[myedge,->] (t1.-165) to[out=-90,in=90] (low);

    \node[transition](t2) at (10mm,0){};
    \draw[myedge,<-] (t2.15) to[out=90,in=-35] (start);
    \draw[myedge,<-] (t2.165) to[out=90,in=0] (phase1);
    \draw[myedge,->] (t2.-165) ..controls +(down:10mm) and +(-80:3mm).. (phase1);
    \draw[myedge,->] (t2.-15) to[out=-90,in=90] (high);

    \node[transition,rotate=90](t3) at (13mm,15mm){};
    \draw[*-,myedge] (t3.15) to[out=180,in=0] (start);
    \draw[myedge,<-] (t3.165) to[out=180,in=35] (phase1);
    \draw[myedge,->] (t3.-90) to[out=0,in=180] (phase2);

    \node[caption] at (0,-15mm) {Arriving at the dock};
  \end{scope}

  %% Going out
  \begin{scope}[shift={(60mm,22mm)}]

    \node[state,place,label={[scale=0.6]right:$High$}](high) at (10mm,0) {};
    \node[state,place,label={[scale=0.6]left:$Low$}](low)  at (-10mm,0) {};
    \node[state,place,label={[scale=0.6,label distance=4pt]above:$Phase2$}](phase2) at (0,-10mm) {};

    \node[state,place,label={[scale=0.6]right:$High_{granted}$}](ghigh) at (10mm,-20mm) {};
    \node[state,place,label={[scale=0.6]left:$Low_{granted}$}](glow)  at (-10mm,-20mm) {};

    \node[state,place,label={[scale=0.6]right:$End$}](finish) at (0,-40mm) {};

    \node[transition](t1) at (-10mm,-10mm){};
    \draw[*-,myedge] (t1.90) to[out=90,in=180] (high);
    \draw[myedge,<-] (t1.165) to[out=90,in=-135] (low);
    \draw[myedge,<-] (t1.15) to[out=90,in=115] (phase2);
    \draw[myedge,->] (t1.-165) to[out=-90,in=90] (glow);
    \draw[myedge,->] (t1.-15) to[out=-90,in=-115] (phase2);

    \node[transition](t2) at (10mm,-10mm){};
    \draw[myedge,<-] (t2.15) to[out=90,in=-35] (high);
    \draw[myedge,<-] (t2.165) to[out=90,in=65] (phase2);
    \draw[myedge,->] (t2.-15) to[out=-90,in=90] (ghigh);
    \draw[myedge,->] (t2.-165) to[out=-90,in=-65] (phase2);

    \node[transition](t3) at (-10mm,-30mm){};
    \draw[myedge,<-] (t3) to[out=90,in=-90] (glow);
    \draw[myedge,->] (t3) to[out=-90,in=135] (finish);

    \node[transition](t4) at (10mm,-30mm){};
    \draw[myedge,<-] (t4) to[out=90,in=-90] (ghigh);
    \draw[myedge,->] (t4) to[out=-90,in=35] (finish);

    \node[caption,left=5mm of finish] {Boarding};
  \end{scope}

  \draw[white,thick] (3,2.5) -- +(0,-4);

\end{tikzpicture}
} %end \resizebox
\par} %end \centering

\fi

\section{Simple Barrier}
\ifnoexperiments\elseA barrier is a tool to ensure synchronization between threads and
allows a programmer to impose restriction for an asynchronous
multi-threaded program.
%
A process arriving at the barrier must stop at this point and cannot
proceed until all other processes reach this barrier.

\begin{wrapfigure}{r}[0pt]{0.30\linewidth}
  \hfill%
  \begin{tikzpicture}
    \node[draw,shade,rotate=90,anchor=center,inner xsep=15mm](barrier) at (0,0){Barrier};
    \foreach \d/\a/\n in {25mm/10/A,13mm/30/B,20mm/150/C,10mm/170/D}{
      \node[circle,draw,shade,left=\d of barrier.\a](\n-1){\n};
      \draw[-latex,dashed,gray] (\n-1) -- (barrier.\a);
    }
    \draw[latex-] (0,-2.3) -- +(-25mm,0) node[fill=white,midway]{time};      
  \end{tikzpicture}
\end{wrapfigure}
%
The following model represents a barrier with a \emph{pivot}. The
first thread at the barrier will take the role of a pivot and all
other processes wait as long as there is a pivot. When all processes
have arrived at the barrier, the pivot can then proceed, which in turn
releases all the waiting processes.

A process can be in state \emph{before}~\w[w]{B},
\emph{wait}~\w[w]{W}, \emph{pivot}~\w[w]{P} or
\emph{after}~\w[w]{A}. %
%
Initially, all processes are in state~\w[i]{B} and a bad configuration
is detected when there is a process in state~\w[w]{A}, while there is
still a process in state~\w[w]{B}.
%
The transition diagram and inhibitor net are as follows.

\begin{wrapfigure}{l}{0.55\linewidth}
\begin{tikzpicture}[%
  show background rectangle,
  transition/.append style={minimum width=6mm},
  mylabel/.append style={scale=0.7},
  every place/.append style={scale=0.7}
  ]

  \draw[white,thick] (-1.1,0) -- +(0,-2);

  %% Diagram
  \begin{scope}[shift={(-2.5,0)}]

    \node[state,state-i](before) at (0,0) {B};
    \node[state,state-n](wait) at (-1,-1) {W};
    \node[state,state-n](pivot) at (1,-1) {P};
    \node[state,state-n](after) at (0,-2) {A};

    \draw[myedge] (before) -- node[mylabel,midway,above left]{$\exists~P$} (wait);
    \draw[myedge] (before) -- node[mylabel,midway,above right]{$\forall~B$} (pivot);

    \draw[myedge] (wait) -- node[mylabel,midway,below left]{$\not\exists~P$} (after);
    \draw[myedge] (pivot) -- node[mylabel,midway,below right]{$\forall~W$} (after);

    %\node[caption] at (0,-1.3) {Arriving at the dock};
  \end{scope}

  %% Petri Net
  \begin{scope}%[xshift=4]

    \node(b)[state,place] at (0,0)  {};
    \node(w)[state,place] at (0,-2) {};
    \node(p)[state,place] at (1,-1) {};
    \node(a)[state,place] at (2,-3) {};

    \node[scale=0.5,left=1pt of b]{B};
    \node[scale=0.5,left=1pt of w] {W};
    \node[scale=0.5,right=1pt of p]{P};
    \node[scale=0.5,right=1pt of a]{A};

    \node[transition](t1) at (0,-1){};
    \draw[myedge,<-] (t1.135) to[out=90,in=-135] (b);
    \draw[myedge,->] (t1.-135) to[out=-90,in=90] (w);
    \draw[myedge,<-] (t1.35) ..controls +(up:6mm) and +(135:6mm).. (p);
    \draw[myedge,->] (t1.-35) ..controls +(down:6mm) and +(-135:6mm).. (p);

    \node[transition](t2) at (1,-2.5){};
    \draw[myedge,<-] (t2.150) ..controls +(up:5mm) and +(right:6mm).. (w);
    \draw[myedge,*-] (t2.30) to[out=90,in=-90] (p);
    \draw[myedge,->] (t2.-90) to[out=-90,in=180] (a);

    \node[transition,rotate=90](t3) at (1.5,0.5){};
    \draw[myedge,<-] (t3.90) to[out=180,in=45] (b);
    \draw[myedge,*-] (t3.160) to[out=180,in=90] (p);
    \draw[myedge,*-] (t3.20) ..controls +(left:25mm) and +(120:25mm).. (w);
    \draw[myedge,->] (t3.-90) to[out=0,in=45] (p);

    \node[transition](t4) at (2,-2){};
    \draw[myedge,*-] (t4.30) to[out=90,in=-10] (b);
    \draw[myedge,<-] (t4.150) to[out=90,in=-45] (p);
    \draw[myedge,->] (t4) to[out=-90,in=90] (a);

    %\node[caption] at (0,-3.3) {Boarding};
  \end{scope}

\end{tikzpicture}
\end{wrapfigure}
%
The following invariant can be derived: %
(i) a pivot~$P$ can coexist with a process in state~\w[w]{B}
and/or \w[w]{W}, or is alone, i.e.\ the set of configurations
$\set{\w[w]{P}\wedge\w[w]{B}^*\wedge\w[w]{W}^*}$, %
(ii) a process after the barrier can only coexist with other process
after the barrier or still waiting, i.e.\ the set of configurations
$\set{\w[w]{W}^*\wedge\w[w]{A}^+}$, %
and (iii) initially, all processes were before the barrier, i.e.\
$\set{\w[w]{B}^*}$.

It is also possible to model it with a linear topology (as
in~\ref{paper:SAS14}).
% Section~\ref{section:introduction}~and~\ref{chapter:parameterized:systems}. %
In that case, we can consider a non-atomic version, which would not
check the state of the other processes all at once.

A typically implementation uses an atomic counter as follows.
\lstinputlisting[style=custom]{experiments/code/barrier.txt}
\fi

\section{List with Counter Automata}
\ifnoexperiments\else%
\begin{wrapfigure}{r}[0pt]{0.34\linewidth}
  \hfill%
  \lstinputlisting[style=custom]{experiments/code/list-counter-automata.txt}
\end{wrapfigure}
%
Consider the following single-threaded program. It scans a list
starting from the pointer $L$ and must end when it finds a marker $P$,
placed prior to the start of the program. %
In this case, the parameter for this parametrized system if the length
of the list.
%
The program risks a segmentation fault is the marker $P$ is not
found. Our analysis retain the presence of the marker information in a
context (and would forget it without context).
\fi

%% =================================================
\section{Michael \& Scott's Lock-Free Queue}
\ifnoexperiments\else\label{MS:lock:free:queue}
%
We show a version of the concurrent queue by Michael and
Scott~\cite{MS:QueueAlgorithms}.
%
The program represents a queue as a linked list from the node pointed
to by \prgcode{Head} to a node that is either pointed by
\prgcode{Tail} or by \prgcode{Tail}'s successor.
%
The global variable \prgcode{Head} always points to a dummy cell whose
successor, if any, stores the head of the queue.
%
In the absence of garbage collection, the program must handle the ABA
problem~\cite{ABA:1983,ABA:Wikipedia} where a thread mistakenly
assumes that a globally accessible pointer has not been changed since
it previously accessed that pointer.
%
Each pointer is therefore equipped with an additional \prgcode{age}
field, which is incremented whenever the pointer is assigned a new
value.

The queue can be accessed by an arbitrary number of threads, either by
an enqueue method \prgcode{enq(d)}, which inserts a cell containing
the data value \prgcode{d} at the tail, or by a dequeue method
\prgcode{deq(d)} which returns \prgcode{empty} if the queue is empty,
and otherwise advances \prgcode{Head}, deallocates the previous dummy
cell and returns the data value stored in the new dummy cell.
%
The algorithm uses the atomic compare-and-swap (CAS) operation.
%
For example, the command %
\prgcode{CAS(\&Head, head, $\langle$next.ptr,head.age+1$\rangle$)} at
line~\ref{ms:code:line:deq} of the \prgcode{deq} method checks whether
the extended pointer \prgcode{Head} equals the extended pointer
\prgcode{head} (meaning that both fields \prgcode{ptr} and
\prgcode{age} must agree).
%
If not, it returns \prgcode{FALSE}. Otherwise it returns
\prgcode{TRUE} after assigning %
\prgcode{$\langle$next.ptr, head.age+1$\rangle$} to \prgcode{Head}.

The linearization points~\commitpoint{} %
are at line~\ref{ms:code:line:enq}, \ref{ms:code:line:empty} and \ref{ms:code:line:deq}.
%
For instance, line~\ref{ms:code:line:enq} of the \prgcode{enq} method
called with data value \prgcode{d} is instrumented to generate the
abstract event \s{\prgcode{enq(d)}} when the \prgcode{CAS} command
succeeds;
%
no abstract event is generated when the \prgcode{CAS} fails.
%
Generation of abstract events can be conditional.
%
For instance, line~\ref{ms:code:line:empty} of the \prgcode{deq}
method is instrumented to generate \s{\prgcode{deq(empty)}} when the
value assigned to \prgcode{next} satisfies \prgcode{next.ptr = NULL}
(i.e.\ it will cause the method to return \prgcode{empty} at
line~\ref{ms:code:line:emptyreturn}).

{\noindent\centering
\begin{tikzpicture}
        
  \node[codeblock] (init) at (current bounding box.north west) {\begingroup\scriptsize\VerbatimInput[numbers=none]{experiments/code/ms/init}\endgroup};
  \node[codeblock] (enq) at (init.south west) {\begingroup\scriptsize\VerbatimInput{experiments/code/ms/enq}\endgroup};
  \node[codeblock] (struct) at (init.north east) {\begingroup\scriptsize\VerbatimInput[numbers=none]{experiments/code/ms/struct}\endgroup};
  \node[codeblock] (deq) at (struct.south west) {\begingroup\scriptsize\VerbatimInput[firstnumber=17]{experiments/code/ms/deq}\endgroup};

  \node[property,rotate=-10,scale=0.7] at ([shift={(-0.5,-0.2)}]init.north east) {\sc Init};
  \node[property,rotate=-10,scale=0.7] at ([shift={(-0.5,-0.2)}]enq.north east) {\sc Enq};
  \node[property,rotate=-10,scale=0.7] at ([shift={(-0.5,-0.2)}]deq.north east) {\sc Deq};

\end{tikzpicture}%
\par%
}
\fi

\section{Treiber's Lock-Free Stack}
\ifnoexperiments\elseYou can find the example for Treiber with Garbage Collection in Section~\ref{section:shape:programs}.
%
We present here the code in the case there is no garbage
collection. As in the previous section, we equip each pointer with an
\prgcode{age} field in order to avoid the ABA problem.

{\noindent\centering
  \begin{tikzpicture}[%
    %codeblock/.append style={text width=48mm},%0.5\textwidth,
    property/.append style={left,rotate=-10,scale=0.7,shift={(-0.5,-0.7)}},
    ]

    \node[codeblock] (struct) {\begingroup\scriptsize\VerbatimInput[numbers=none]{experiments/code/treiber/struct}\endgroup};
    \node[codeblock] (init) at (struct.north east) {\begingroup\scriptsize\VerbatimInput[numbers=none]{experiments/code/treiber/init}\endgroup};
    \node[codeblock] (pop) at (init.south west) {\begingroup\scriptsize\VerbatimInput[firstnumber=10]{experiments/code/treiber/pop}\endgroup};
    \node[codeblock,above left] (push) at (pop.south west) {\begingroup\scriptsize\VerbatimInput{experiments/code/treiber/push}\endgroup};
    
    \node[property] at (init.north east) {\sc Init};
    \node[property] at (push.north east) {\sc Push};
    \node[property] at (pop.north east) {\sc Pop};
    
  \end{tikzpicture}
\par%
}
\fi

% %% =================================================
% % Adds back the sections to the Table of Content, in case.
% \addtocontents{toc}{\protect\setcounter{tocdepth}{1}}
