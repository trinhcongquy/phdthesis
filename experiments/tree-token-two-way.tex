\index{Two-way Token}

This protocol is a generalization of the token protocol by allowing
the token to both move upwards and downwards.
%
Initially, the token is in the root.
%
The set of bad constraints \Bad\ is represented by trees where at
least two nodes contain the token, similarly to the simple token
protocol.
%
The rules for the propagation of the token are:

{\noindent\centering
  \begin{tikzpicture}[show background rectangle, sibling distance=10mm,level distance=12mm,
    every node/.style={state,state-n,inner sep=2pt,solid} ]
    \node {t / n} child {node {n / t}} child[missing];
  \end{tikzpicture}
  \begin{tikzpicture}[show background rectangle, sibling distance=10mm,level distance=12mm,
    every node/.style={state,state-n,inner sep=2pt,solid} ]
    \node {t / n} child[missing] child {node {n / t}};
  \end{tikzpicture}
  \begin{tikzpicture}[show background rectangle, sibling distance=10mm,level distance=12mm,
    every node/.style={state,state-n,inner sep=2pt,solid} ]
    \node {n / t} child {node {t / n}} child[missing];
  \end{tikzpicture}
  \begin{tikzpicture}[show background rectangle, sibling distance=10mm,level distance=12mm,
    every node/.style={state,state-n,inner sep=2pt,solid} ]
    \node {n / t} child[missing] child {node {t / n}};
  \end{tikzpicture}
  \par%
}

\newpage
As an example:

{\noindent\centering
  \begin{tikzpicture}[show background rectangle,
    level 1/.style={sibling distance=10mm,level distance=12mm},
    level 2/.style={sibling distance=5mm,level distance=12mm},
    every node/.style={state,state-n,inner sep=2pt,solid},
    ]
    \node(t1)[red] {t}
    child {node {n}
      child {node {n}}
      child {node {n}}}
    child {node {n}
      child {node {n}}
      child {node {n}}
    };
    \node(t2)[xshift=30mm] {n}
    child {node[red] {t}
      child {node {n}}
      child {node {n}}}
    child {node {u}
      child {node {n}}
      child {node {n}}
    };
    \node(t3)[xshift=60mm] {n}
    child {node {n}
      child {node {n}}
      child {node[red] {t}}}
    child {node {n}
      child {node {n}}
      child {node {n}}
    };
    \node(t4)[xshift=90mm] {n}
    child {node[red] {t}
      child {node {n}}
      child {node {n}}}
    child {node {u}
      child {node {n}}
      child {node {n}}
    };
    \node(t5)[xshift=120mm] {n}
    child {node {n}
      child[red] {node {t}}
      child {node {n}}}
    child {node {n}
      child {node {n}}
      child {node {n}}
    };

    %% Small step arrows
    \foreach \f/\t in {1/2,2/3,3/4,4/5}{
      \path (t\f) -- coordinate[midway](t\f\t) (t\t);
      \draw[*->] (t\f\t) ++(-3mm,0) .. controls +(3mm,1mm) .. +(6mm,0);
    }
  \end{tikzpicture}
  \par%
}

