\index{Firewire}

The 1394 High Performance serial bus~\cite{ieee:1394:firewire} is used
to transport digitized video and audio signals within a network of
multimedia systems and devices.
% 

The tree identification protocol is used in one of the phases
implementing the IEEE 1394 protocol.
%
More precisely, it is run after a bus reset in the network and leads
to the election of a unique leader node.

In this section, we consider a version working on tree topologies.
%
Furthermore, we assume that (i) each inner node is connected to $3$
neighbors, (ii) the root is connected to $2$ neighbors, and (ii)
communication is atomic.

Initially, all nodes are in state \emph{undefined}~\s{u}. 
%
We identify two steps in the protocol depending on the number~$n$ of
neighbors which are still in state~\s{u}.
%
If $n>1$, the node waits for (``be my parent'') requests from its neighbors. 
%
If $n=1$, the node sends a request to the remaining neighbor in state~\s{u}. 
%
We can observe that the leaf nodes are the first to communicate with
their neighbors.

Formally, we derive a parameterized tree system model $\parsys$ as follows.
%
We define the set of states by
$\locs=\setcomp{\s{\ensuremath{s_n}}}{s\in\set{\s{u},\s{c},\s{\ensuremath{\ell}}}\wedge
  n\in\set{leaf,inner,root}}$
where the script $n$ describe the nature of the node.
%
In the definition of the state ($s$), the letters \s{u}, \s{c} and
\s{$\ell$} stand respectively for \emph{undefined}, \emph{child} and
\emph{leader}.
%
In a similar manner to the previous protocol, we drop the script
whenever we mean that it can take any value (see caption of
Figure~\ref{figure:tree:arbiter:rules}).
% 

The rewrite rules $\rules$ are described as follows.
%
\begin{itemize}
\item The leaves initiate the communications:

{\noindent\centering
  \begin{tikzpicture}[show background rectangle,
    level 1/.style={sibling distance=15mm,level distance=10mm},
    every node/.style={state,state-n,inner sep=2pt}
    ]
    \node {u} child{node{u$_{leaf}$ / c}} child[missing];
  \end{tikzpicture}
  % 
  \begin{tikzpicture}[show background rectangle,
    level 1/.style={sibling distance=15mm,level distance=10mm},
    every node/.style={state,state-n,inner sep=2pt}
    ]
    \node {u} child[missing] child{node{u$_{leaf}$ / c}};
  \end{tikzpicture}
  \par%
}

\item The inner nodes become children or wait for requests:

{\noindent\centering
  \begin{tikzpicture}[show background rectangle,
    grow cyclic,
    level 1/.style={level distance=1cm,sibling angle=120},
    every node/.style={state,state-n,inner sep=2pt}
    ]
    \node {u / c} child {node {u}} child {node {c}} child {node {c}};
  \end{tikzpicture}
  % 
  \begin{tikzpicture}[show background rectangle,
    grow cyclic,
    level 1/.style={level distance=1cm,sibling angle=120},
    every node/.style={state,state-n,inner sep=2pt}
    ]
    \node {u / c} child {node {c}} child {node {u}} child {node {c}};
  \end{tikzpicture}
  % 
  \begin{tikzpicture}[show background rectangle,
    grow cyclic,
    level 1/.style={level distance=1cm,sibling angle=120},
    every node/.style={state,state-n,inner sep=2pt}
    ]
    \node {u / c} child {node {c}} child {node {c}} child {node {u}};
  \end{tikzpicture}
  \par%
}

\item The leader is chosen:

{\noindent\centering
  \begin{tikzpicture}[show background rectangle,
    grow cyclic,
    level 1/.style={level distance=1cm,sibling angle=120},
    every node/.style={state,state-n,inner sep=2pt}
    ]
    \node {u / $\ell$} child {node {c}} child {node {c}} child {node {c}};
  \end{tikzpicture}
  % 
  \begin{tikzpicture}[show background rectangle,
    grow cyclic,
    level 1/.style={level distance=1cm,sibling angle=120},
    every node/.style={state,state-n,inner sep=2pt}
    ]
    \node {c} child{node {u$_{leaf}$ / $\ell$}} child[missing] child[missing];
  \end{tikzpicture}
  % 
  \begin{tikzpicture}[show background rectangle,
    grow cyclic,
    level 1/.style={level distance=1cm,sibling angle=120},
    every node/.style={state,state-n,inner sep=2pt}
    ]
    \begin{scope}[rotate=60]
      \node {c} child{node {u$_{leaf}$ / $\ell$}} child[missing] child[missing];
    \end{scope}
  \end{tikzpicture}
  \par%
}
\end{itemize}
%

The set of initial configurations \Inits\ is represented by trees where all nodes
are in state undefined, and the set of bad constraints \Bad\ is represented by 
trees where at least 2 leaders are elected.

{\noindent\centering
  \begin{tikzpicture}[show background rectangle, level distance=1cm, every node/.style={state,state-n,inner sep=2pt}]
    \node {$\ell$} child {node {$\ell$}} child[missing];
  \end{tikzpicture}
  % 
  \begin{tikzpicture}[show background rectangle, level distance=1cm, every node/.style={state,state-n,inner sep=2pt}]
    \node {$\ell$} child[missing] child {node {$\ell$}};
  \end{tikzpicture}
  % 
  \begin{tikzpicture}[show background rectangle, level distance=1cm, every node/.style={state,state-n,inner sep=2pt}]
    \node {*} child {node {$\ell$}} child {node {$\ell$}};
  \end{tikzpicture}
  \par%
}
