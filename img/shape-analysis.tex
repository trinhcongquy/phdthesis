\begin{tikzpicture}

  %% ============ Concrete shape ======================
  \begin{scope}[scale=0.8,every node/.append style={scale=0.8}]

    \foreach \n/\x/\y/\c in {
      1/0/0/white,
      2/1/-3/white,
      3/2/-1.5/red,
      4/3/-4/white,
      5/0.5/-5.5/white,
      6/3.5/-5.5/blue}{
      
      \node(m\n)[minimum width=12mm,minimum height=6mm,inner sep=0pt,
      fill=blue!20, draw=blue!50, very thick, rounded corners=4pt] at (\x,\y) {};
      \node[data=\c,left=1mm of m\n.center]{}; 
      \coordinate[right=2mm of m\n.center](m\n-n){};
      \draw[blue!50, very thick] (m\n.north) -- (m\n.south);
    }
    
    \node[globals](null) at (3,-7){\#};
  
    \draw[pointsto] (m1-n) .. controls +(right:20mm) and +(160:30mm) .. (m2);
    \draw[pointsto] (m2-n) .. controls +(right:20mm) and +(-170:20mm) .. (m3);
    \draw[pointsto] (m3-n) .. controls +(right:20mm) and +(150:20mm) .. (m4);
    \draw[pointsto] (m4-n) .. controls +(-15:32mm) and +(160:25mm) .. (m5);
    \draw[pointsto] (m5-n) .. controls +(right:10mm) and +(120:10mm) .. (null);
    \draw[pointsto] (m6-n) .. controls +(-10:20mm) and +(50:10mm) .. (null);


    %% ============ Variables ======================

    \node[globals,above left=3pt of m1.west,xshift=-2pt] (T) {Top};
    \node[thread1,below left=3pt of m1.west,xshift=-3pt] (t1) {t};
    \node[thread1,left=2pt of m2.west,xshift=-2pt] (x1) {x};
    \node[thread2,left=2pt of m4.west,xshift=-2pt] (t2) {t};
    \node[thread2,left=1pt of m6.west,xshift=-2pt] (n2) {n};

    %% ============ Threads info ======================
    \node[fill=t1color,rectangle,inner xsep=2pt] (thread1) at (3.2,0.2) {Thread 1 {\tiny (pc:~\ref{treiber:thread:pop})}};
    \node[fill=t2color,rectangle,inner xsep=2pt,below=2pt of thread1.south] (thread2) {Thread 2 {\tiny (pc:~\ref{treiber:thread:push})}};

    
    %% ============ White separation ======================
    \draw[white,ultra thick] (5.25,0) -- +(0,-7);

  \end{scope} %% End concrete shape

  %% ============ Matrix Encoding ===============
  \begin{scope}[shift={(4.5,-0.5)},scale=0.8,every node/.append style={scale=0.8}]

    \matrix[matrix of math nodes, below right, inner sep=0pt, nodes={var,inner sep=0pt}, row sep=0pt,column sep=0pt] (encoding) {
      & |[globals]|\text{Top}
      & |[globals]|\text{\#}
      & |[thread1]|\text{t}
      & |[thread1]|\text{x}
      & |[thread2]|\text{t}
      & |[thread2]|\text{n}
      & |[data=red,minimum size=1em]|
      & |[data=blue,minimum size=1em]|
      \\
      |[globals]|\text{Top}
      & =          & \reaches  & =           & \pointsto  & \reaches   & \unrelated & \reaches   & \unrelated \\
      |[globals]|\text{\#} 
      & \reachedby & =         & \reachedby  & \reachedby & \reachedby & \pointedby & \reachedby & \pointedby \\
      |[thread1]|\text{t} 
      & =          & \reaches  & =           & \pointsto  & \reaches   & \unrelated & \reaches   & \unrelated \\
      |[thread1]|\text{x} 
      & \pointedby & \reaches  & \pointedby  & =          & \reaches   & \unrelated & \pointsto  & \unrelated \\
      |[thread2]|\text{t} 
      & \reachedby & \reaches  & \reachedby  & \reachedby & =          & \unrelated & \pointedby & \unrelated \\
      |[thread2]|\text{n}
      & \unrelated & \pointsto & \unrelated  & \unrelated & \unrelated & =          & \unrelated & =          \\
      |[data=red,minimum size=1em]|
      & \reachedby & \reaches  & \reachedby  & \pointedby & \pointsto  & \unrelated & =          & \unrelated \\
      |[data=blue,minimum size=1em]|
      & \unrelated & \pointsto & \unrelated  & \unrelated & \unrelated & =          & \unrelated & =          \\
    };

    
  \draw (encoding-2-2.north west) -- (encoding-2-9.north east);
  \draw (encoding-2-2.north west) -- (encoding-9-2.south west);
  \foreach \n in {2,...,9}{
    \draw (encoding-\n-2.south west) -- (encoding-\n-9.south east);
    \draw (encoding-2-\n.north east) -- (encoding-9-\n.south east);
  }

  \begin{pgfonlayer}{my background}
    \fill[green!5]  (encoding-2-2.north west) rectangle (encoding-3-3.south east);
    \fill[yellow!5] (encoding-2-4.north west) rectangle (encoding-3-5.south east);
    \fill[yellow!5] (encoding-4-2.north west) rectangle (encoding-5-5.south east);

    \fill[pink!25] (encoding-2-6.north west) rectangle (encoding-3-7.south east);
    \fill[pink!25] (encoding-6-2.north west) rectangle (encoding-7-3.south east);
    \fill[pink!25] (encoding-6-6.north west) rectangle (encoding-7-7.south east);

    \fill[orange!10] (encoding-4-6.north west) rectangle (encoding-5-7.south east);
    \fill[orange!10] (encoding-6-4.north west) rectangle (encoding-7-5.south east);
  \end{pgfonlayer}


  %% ============ Key ======================
  \matrix[matrix of nodes,nodes={inner xsep=2pt},above=5pt of encoding] (sigma) {
    $\sigma$ & : &
    |[fill=t1color,rectangle]| \ref{treiber:thread:pop} &
    $|$ &
    |[fill=t2color,rectangle]| \ref{treiber:thread:push} &
    $|$ &
    Observer: $s_1$ \\    
  };


  %% ============ Pi ======================
  \node[inner sep=0pt,right=8pt of encoding.east,scale=0.8] (pi){$\pi[t_2,n_2]$};
  \begin{pgfonlayer}{my background}
    \begin{scope}
      [bubble/.style={fill=purple!30!white}]%opacity=.3,transparency group % Note: transparency doesn't work with xetex
      \node [bubble,circle,fit=(encoding-6-7.center)] (pit2n2){};
      \node [bubble,ellipse,fit=(pi)] (pit2n2'){};
      
      \fill[bubble]
      (pit2n2.north) .. controls +(right:3mm) and +(-50:3mm) .. (pit2n2'.-170) --
      (pit2n2'.south) .. controls +(left:3mm) and +(up:3mm) ..  (pit2n2.east) -- cycle;
    \end{scope}
  \end{pgfonlayer}

      
  \end{scope} %% End matrix encoding

  \begin{pgfonlayer}{background}
    \coordinate[right=3pt of encoding.east](cheat);
    \node [background rectangle,fit=(T)(null)(encoding)(sigma)(cheat),inner xsep=8pt] (bg){};
  \end{pgfonlayer}
  
\end{tikzpicture}
