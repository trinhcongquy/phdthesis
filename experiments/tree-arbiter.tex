\index{Arbiter}\index{Tree-Arbiter Protocol}

The protocol supervises the access to a shared resource of a set of
processes arranged in a tree topology.
%
The processes competing for the resource reside in the leaves.
%
A process in the protocol can be in state \emph{idle}~\s{i},
\emph{requesting}~\s{r}, \emph{token}~\s{t} or
\emph{below}~\s{b}.
%
All the processes are initially in state~\s{i}.
%
A node is in state~\s{b} whenever it has a descendant in
state~\s{t}.
%
When a leaf is in state~\s{r}, the request is propagated upwards
until it encounters a node which is aware of the presence of the token
(i.e. a node in state~\s{t} or~\s{b}).
%
A node that has the token (in state~\s{t}) can choose to pass it
upwards or pass it downwards to a requesting child (node in
state~\s{r}).
%

We model the tree-arbiter protocol with a parameterized tree system
$\parsys=(\locs,\rules)$ where
$\locs=\setcomp{\s{\ensuremath{s_n}}}{s\in\set{\s{i},\s{r},\s{t},\s{b}}\wedge
  n\in\set{leaf,inner,root}}$
and $\rules$ is as depicted below, in Figure~\ref{figure:tree:arbiter:rules}.
%
Observe that in the definition of $\locs$, we use the subscript~$n$ to
model the nature (leaf, inner or root) of the nodes.
%
In the definition of the rules, we will drop the script whenever we
mean that it is arbitrary (it can take any value).
%

The rules to model this protocol are as follows: 
%
2 rules to propagate the request upwards,
%
2 rules to propagate the token downwards, 
%
2 rules to propagate the token upwards and one rule to initiate a request from a leaf. 
%

\begin{figure}[hb]
  \centering
  \begin{tikzpicture}[show background rectangle,
    level 1/.style={sibling distance=15mm,level distance=10mm},
    every node/.style={state,state-n,inner sep=2pt}
    ]

    \node(t1)                    {i/r} child[missing] child {node {r}};
    \node(t2)[right=20mm of t1]  {i/r} child {node {r}} child[missing];
    \node(t3)[right= 5mm of t2]  {t/b} child[missing] child {node {r/t}};
    \node(t4)[right=20mm of t3]  {t/b} child {node {r/t}} child[missing];
    \node(t5)[right= 5mm of t4]  {b/t} child[missing] child {node {t/i}};
    \node(t6)[right=20mm of t5]  {b/t} child {node {t/i}} child[missing];
    \coordinate[right=8mm of t6](t7); % cheating
    \node[below=5mm of t7]  {i$_{leaf}$/r};

    %\foreach \f/\t in {1/2,2/3,3/4,4/5,5/6}{
    \foreach \f/\t in {2/3,4/5}{
      \path (t\f) -- coordinate[midway](t\f\t) (t\t);
      \draw[color=white] (t\f\t) -- +(0,-10mm); % vertical
    }
    \draw[color=white] (t6.east) ++(2mm,0) -- +(0,-10mm); % vertical t67
  \end{tikzpicture}
  \label{figure:tree:arbiter:rules}
  \caption{The rewrite rules for the tree-arbiter protocol.
    % 
    Notice that there are more rules in the model:
    % 
    For example, the first rule on the left is represented in the concrete model by $2$ rules, each of which 
    corresponds to a particular combination of the natures of the parent and child nodes:
    % 
    For the parent there are $2$ possibilities
    (\s{$i_{inner}/r_{inner}$} and \s{$i_{root}/r_{root}$}) while for
    the child, there are $2$ (\s{$r_{inner}$} and \s{$r_{leaf}$}).}
\end{figure}

\begin{wrapfigure}{r}{0.3\linewidth}
  \resizebox{\linewidth}{!}{%
  \begin{tikzpicture}[show background rectangle,
    level 1/.style={sibling distance=15mm,level distance=10mm},
    every node/.style={state,state-n,inner sep=2pt}
    ]
    \node {*} child {node {$t_{leaf}$}} child {node {$t_{leaf}$}};
  \end{tikzpicture}
  } % end resizebox
\end{wrapfigure}
%
The set of initial configurations \Inits\ contains all trees where
the leaf nodes are either idle or requesting, inner nodes are idle,
and the root has the token.
%
The set of bad constraints \Bad\ is represented by trees where at
least two processes obtained the token (i.e. two leaves in
state~\s{$t_{leaf}$}, see the figure on the right).
