\label{app:dijkstra}
Dijsktra's algorithm %\cite{mutex:dijsktra}
implements a mutual exclusion protocol and can be modeled as a
parameterized system where the processes are arranged in a linear
topology. Each process can communicate with its neighbors and check
their status.

\medskip{\noindent\centering
  \begin{minipage}{0.7\linewidth}
    \lstinputlisting[style=custom]{experiments/code/dijkstra.txt}
  \end{minipage}%
  \par%
}\medskip
%
The algorithm is described in the above code listing.
%
It makes use of a pointer, i.e. a variable ranging over
process indices. We model this pointer by a local boolean variable $p$
for each process state.
%
$p$ is $\strue$ iff the pointer points to this current process. When
the pointer changes, this information must be passed onto all other
processes, which we model as a broadcast transition. Concretely, upon
pointer assignment, the current process sets its local variable $p$ to
$\strue$ and simultaneously sets $p$ to $\sfalse$ in all other
processes.

We denote the state of process as $St$ (resp. $Sf$) when the process
is in state $S$ and the pointer $p$ is $\strue$ (resp. $\sfalse$). The
state $S$ of a process ranges over $\set{\w[w]{1},\ldots,\w[w]{6}}$
where $\w[w]{6}$ represents the critical section.
%
Initially, one process is in state $\w[i]{1t}$ and all other processes
are in state $\w[i]{1f}$. A bad configuration is detected when 2 or
more processes are in the critical section, ie when their state is
either $\w[b]{6t}$ or $\w[b]{6f}$.

% \begin{figure}[h]
\noindent%
\resizebox{\linewidth}{!}{%
\begin{tikzpicture}[%
  node distance=20mm,
  state/.append style={scale=1.25},
  ]

  \node[state,state-i] (n1t) {1f};
  \node[state,state-n] (n2t) [right=of n1t] {2f};
  \node[state,state-n] (n3t) [right=of n2t] {3f};
  \node[state,state-n] (n4t) [right=of n3t] {4f};
  \node[state,state-n] (n5t) [right=of n4t] {5f};
  \node[state,state-b] (n6t) [right=of n5t] {6f};

  \node[state,state-i] (n1f) [below=of n1t] {1t};
  \node[state,state-n] (n2f) [right=of n1f] {2t};
  \node[state,state-n] (n3f) [right=of n2f] {3t};
  \node[state,state-n] (n4f) [right=of n3f] {4t};
  \node[state,state-n] (n5f) [right=of n4f] {5t};
  \node[state,state-b] (n6f) [right=of n5f] {6t};

  \draw [->,myedge] (n1t) -- (n2t);
  \draw [->,myedge] (n2t) -- node[trlabel]{\begin{tabular}{c}$\forall j\neq i :$\\$\set{[1-6]t,1f}$\end{tabular}} (n3t);
  \draw [->,myedge] (n4t) -- (n5t);
  \draw [->,myedge] (n5t) -- node[trlabel]{\begin{tabular}{c}$\forall j\neq i :$\\$\set{[1-4]\{t,f\}}$\end{tabular}} (n6t);
  \draw [->,myedge] (n5t) .. controls +(-1,1) and +(1,1) .. node[trlabel,above]{$\exists j\neq i :\set{5t,6t,5f,6f}$} (n1t);
  \draw [->,myedge] (n6t) .. controls +(0,2) and +(0,2) .. (n1t);


  \draw [->,myedge] (n3t) -- node[trlabel,sloped,above]{\emph{Broadcast}} (n4f);

  \draw [->,myedge] (n1f) -- (n2f);
  \draw [->,myedge] (n2f) to[out=-20,in=-160] (n4f);

  \draw [->,myedge] (n4f) -- (n5f);
  \draw [->,myedge] (n5f) -- node[trlabel]{\begin{tabular}{c}$\forall  j\neq i : $\\$\set{[1-4]\{t,f\}}$\end{tabular}} (n6f);
  \draw [->,myedge] (n5f) .. controls +(-1,-1) and +(1,-1) .. node[trlabel,below]{$\exists j\neq i : \set{5t,6t,5f,6f}$} (n1f);

  \draw [->,myedge] (n6f) .. controls +(0,-2) and +(0,-2) .. (n1f);

  \foreach \x in {1,2,3,4,5,6}{ \draw [<->,myedge,draw=gray] (n\x t) -- node[rotate=45,scale=0.5,midway]{\color{gray}Global change} (n\x f); }     

  %% Uncomment to see the limits of the bounding box
  % \fill[red] (current bounding box.north west) circle (2pt);
  % \fill[red] (current bounding box.south east) circle (2pt);
\end{tikzpicture}
} % End resise
% \caption{The transitions of Dijkstra's protocol}
% \label{figure:dijkstra:transitions}
% \end{figure}
