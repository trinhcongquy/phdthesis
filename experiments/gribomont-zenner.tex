This algorithm could be seen as a version of Szymanski's
algorithm~\ref{figure:szymanski:implementation}, with transitions that
are finer-grained in the sense that tests and assignments are split
over different atomic transitions.
%
In this model, the local state of a process ranges over
state~$\set{\w[w]{1},\ldots,\w[w]{13}}$ where \w[i]{1} is the initial
state and \w[b]{12} represents the critical section.  Configurations
not satisfying mutual exclusion are those where at least two processes
are at state~\w[b]{12}.

{\noindent%
  %\centering
  \resizebox{\linewidth}{!}{%
  \begin{tikzpicture}[%
        node distance=25mm,
        state/.append style={minimum width=6mm}%
  ]     

  \node[state,state-i] (n1) {1};
  \node[state,state-n] (n2) [left=10mm of n1] {2};
  \node[state,state-n] (n3) [left=10mm of n2] {3};
  \node[state,state-n] (n4) [below=10mm of n3] {4};
  \node[state,state-n] (n5) [below=10mm of n4] {5};
  \node[state,state-n] (n6) [right=of n5] {6};
  \node[state,state-n] (n7) [right=of n6] {7};
  \node[state,state-n] (n8) [right=of n7] {8};
  \node[state,state-n] (n9) [right=of n8] {9};
  \node[state,state-n] (n10) [above=10mm of n9] {10};
  \node[state,state-n] (n11) [above=10mm of n10] {11};
  \node[state,state-b] (n12) [left=of n11] {12};
  \node[state,state-n] (n13) [left=6mm of n12] {13};

   \draw [->,myedge] (n1) -- (n2);
   \draw [->,myedge] (n2) -- (n3);
   \draw [->,myedge] (n3) -- node[trlabel,right]{$\forall j\neq i : \set{1,2,3,4,7,8}$} (n4);
   \draw [->,myedge] (n4) -- (n5);
   \draw [->,myedge] (n5) -- node[trlabel,above]{$\exists j\neq i : \set{3,4,10-13}$} (n6);
   \draw [->,myedge] (n6) -- (n7);
   \draw [->,myedge] (n7) -- node[trlabel,above]{$\exists j\neq i : \set{10-13}$} (n8);
   \draw [->,myedge] (n8) -- (n9);
   \draw [->,myedge] (n9) -- (n10);
   \draw [->,myedge] (n10) -- node[trlabel,left]{$\forall j\neq i: \set{1-4,10-13}$} (n11);
   \draw [->,myedge] (n11) -- node[trlabel,above]{$\forall j<i: \set{1-4,7,8}$} (n12);

   \draw [->,myedge] (n12) -- (n13);
   \draw [->,myedge] (n13) -- (n1);

   %\draw [->,myedge] (n5) to[out=-30,in=-150] node[mylabel,above,pos=0.5]{$\forall j\neq i : \set{1,2,5-9}$} (n9);
   \draw [->,myedge] (n5) ..controls +(-45:10mm) and +(-135:10mm).. (n9) node[trlabel,below,pos=0.5]{$\forall j\neq i : \set{1,2,5-9}$};

\end{tikzpicture}%
}% End resize
%\par%
}
%\def\initstate{\tikz[baseline=(n.base)]\node[state,fill=green!20,scale=0.7](n){1};}
%\caption{A pseudocode of the $i$th process of
%  Gribomont-Zenner's protocol and the corresponding transition rules (in the form of a transition diagram). 
%  Initially, all processes are in state {\protect\initstate}.
%}
%\caption{Transition rules of Gribomont-Zenner's protocol.}
%
